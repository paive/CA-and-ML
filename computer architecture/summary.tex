\documentclass[a4paper]{ctexart}

\usepackage[top=1in, bottom=1in, left=1in, right=1in]{geometry}
\usepackage{titlesec}
\usepackage{amsmath}
\usepackage{ulem}
\newcommand{\li}{\uline{\hspace{0.5em}}}

\begin{document}

\title{高级体系结构知识点}
\author{杨森}
\maketitle

\section{概述}
\subsection{名词解释}
\begin{enumerate}
  \item \emph{\textbf{计算机体系结构}}:计算机系统结构是指传统机器程序员看到的计算机属性,即概念性结构和功能特性。(2001)
  
\end{enumerate}

\section{指令系统}
\subsection{名词解释}
\begin{enumerate}
  \item \emph{\textbf{通用寄存器型机器}}:指令系统为通用寄存器型结构的机器。(2001)
  \item 指令系统的\emph{\textbf{规整性}}:指令系统的规整性主要包括对称性和均匀性,对称性是指所有与指令系统有关的存储单元的使用、操作码的设置等都是对称的;均匀性对于各种不同的操作数类型、字长、操作种类和数据存储单元,指令的设置都要同等对待。(2001)
\end{enumerate}

\section{流水线}
\subsection{名词解释}
\begin{enumerate}
  \item \emph{\textbf{结构相关}}:在流水线处理机中,如果某种指令组合因为资源冲突而不能正常执行,则称该处理机有结构相关。(2001)
\end{enumerate}


\section{向量处理机}

\section{指令级并行}
\subsection{名词解释}
\begin{enumerate}
  \item \emph{\textbf{乱序流出(Out of order issue)}}:指令的流出顺序与程序顺序不同。(2001)
  \item \emph{乱序执行(Out of order Execution)}:指令的执行顺序与程序顺序不同。
  \item \emph{乱序完成(Out of order completion)}:指令的完成顺序与程序顺序不同。
  \item \emph{\textbf{再定序缓冲(ReOrder Buffer)}}:暂存指令执行的结果,使其不直接写回到寄存器或者存储器,能够在分支错误的情况下恢复现场。(2001)
\end{enumerate}

\section{存储系统}
\subsection{名词解释}
\begin{enumerate}
  \item \emph{\textbf{虚拟cache}}:虚拟Cache是指可以直接用虚拟地址进行访问的Cache,其标识存储器中存放的是虚拟地址,进行地址检测用的也是虚拟地址。(2001)
  \item \emph{\textbf{存储体冲突}}:两个访问请求要求访问同一个存储体。(2001)
\end{enumerate}

\section{互连网络}

\begin{enumerate}
  \item 在拓扑上,互连网络为输入和输出两组节点之间提供了一组\emph{\underline{互连}}或\emph{\underline{映像}}。
  \item \textbf{静态互连网络}:由点和点直接相连而成
  \item \textbf{动态互连网络}:由\underline{\emph{开关通道}}实现,可以动态改变结构
  \item \textbf{节点度}:与节点相连接的边的数目称为节点度,这里的边表示链路或通道,进入节点的通道数为入度,从节点出来的通道数为出度(\emph{节点度=入度+出度})。
  \item \textbf{网络直径}:网络中任意两个节点间最短路径长度的最大值称为\underline{\emph{网络直径}}。
  \item \textbf{等分宽度}:将网络切成任意相等两半的各种切法中,沿切口的最小通道边数称为\underline{\emph{通道等分宽度}}
  \item 对于一个网络,如果从任何一个节点看,拓扑结构都一样,则称此网络为对称网络。
  \item \textbf{路由}:在网络通信中对路径的选择与指定。
\end{enumerate}

\section{多处理机}
\subsection{同步}
\begin{enumerate}
  \item 基本硬件原语:\emph{原子交换(Atomic Exchange)}将一个存储单元的值和一个寄存器的值进行交换;\emph{测试并置定(test\li and\li set)}先测试一个存储单元的值,如果符合条件则修改其值;\emph{读取并加1(fetch\li and\li increment)}返回存储单元的值并自动增加该值;\emph{指令对LL/SC},从第二条指令的返回值判断该指令执行是否成功,这两条指令之间不能插入其他对存储单元进行操作的其他指令。

前三种硬件原语都要执行两次访存操作。LL/SC可以用于实现其他同步原语
例:用LL/SC实现原子交换
\begin{equation}
\begin{split}
  try:\ &OR \qquad R3, R4, R0 \\
  &LL  \qquad R2, 0(R1) \\
  &SC \qquad R3, 0(R1) \\
  &BNQZ  \quad R3, try \\
  &MOV \quad R4, R2
\end{split}
\end{equation}

\item \emph{旋转锁(Spin Locks)}:处理器不停地请求获得使用权的锁(锁占用时间少,且加锁过程延迟很低时可采用旋转锁)
\item 使用Cache一致性实现锁的好处:1.可使“环绕”进程(对锁不停的进行测试和请求占用的循环过程)只对本地Cache中的锁(副本)进行操作;2.可利用所访问的局部性,即处理器最近使用过的锁不久又会使用。

\item \emph{\underline{栅栏同步}},栅栏强制所有的到达该栅栏的进程进行等待,直到全部的进程到达栅栏,然后释放全部的进程,从而形成同步。

栅栏的实现需要两个旋转锁,一个用来记录到达栅栏的进程数,另一个用来封锁进程直至最后一个进程到达栅栏。
\underline{栅栏的一个缺陷}:假设有一个进程还没有离开栅栏,即停留在旋转锁等待操作上。如果另外一个比较快的进程又到达了栅栏,而上一次循环的进程中最后那个还没来得及离开栅栏。那么这个“快”进程就会吧release置为0,从而把上次循环的“慢”进程捆到这个栅栏上,这样所有的进程在这个栅栏上的又一次使用都会处于无线等待状态,因为已经到达栅栏的进程数目永远不会等于total。

\end{enumerate}

\end{document}
