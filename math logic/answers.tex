\documentclass[a4paper]{ctexart}

\usepackage[top=1in, bottom=1in, left=1in, right=1in]{geometry}
\usepackage{titlesec}
\usepackage{amsmath}
\usepackage{amsthm}
\usepackage{amssymb}
\usepackage{multirow}
\usepackage{ulem}
\newcommand{\li}{\uline{\hspace{0.5em}}}
\renewcommand{\labelenumi}{\arabic{enumi})}

\begin{document}

\title{数理逻辑课后题/考题答案}
\author{作者:杨森}
\maketitle
\newpage

\section{课后题}
\noindent 1.略

\noindent 2.解:
\begin{enumerate}
  \item =>
  利用结构归纳法证明
  \begin{enumerate}
    \item Y是命题变元,此时Y的生成序列即为自身;
    \item $Y=\sim A$,A的生成序列为$A_1,A_2,\cdots,A_m(=A)$,则Y的生成序列为$A_1,A_2,\cdots,A_m,Y$;
    \item $Y=B\vee C$,B的生成序列为$B_1,B_2,\cdots,B_m(=B)$,C的生成序列为$C_1,C_2,\cdots,C_n(=C)$,则Y的生成序列为$B_1,B_2,\cdots,B_m,C_1,C_2,\cdots,C_n,Y=(B\vee C)$
  \end{enumerate}
  
  \item <= 
  利用第二归纳法证明
  
  假设Y的生成序列为$Y_1,Y_2,\cdots,Y_m(=Y)$,证明$Y_i(1\leq i\leq m)$是合式公式
  \begin{enumerate}
    \item $Y_i$是命题变元,则$Y_i$是合适公式;
    \item $Y_i=\sim Y_j(j<i)$,因为$Y_j$是合式公式,故$Y_i$也是合式公式;
    \item $Y_i=Y_j \vee Y_k(j,k<i)$,因为$Y_j$,$Y_k$是合式公式,故$Y_i$也是合式公式。
  \end{enumerate}
\end{enumerate}
综上,Y为合式公式当且仅当Y有一个生成序列。\newline
\noindent 3.略(根据公式定义进行证明)

\noindent 4.略

\noindent 5.解:
用结构归纳法证明
\begin{enumerate}
  \item A为命题变元p,显然结论成立;
  \item $A=\sim B$,因为B满足条件,则$\sim (B)$即A也满足条件;
  \item $A=B\vee C$,因为B,C满足条件,则$(B)\vee (C)$也满足条件
\end{enumerate}
综上,若表达式A为合式公式,则最终计数为0.\newline
\noindent 6.略

\noindent 7.略

\noindent 8.解:
用公式结构归纳法证明
\begin{enumerate}
  \item A为命题变元p
  \begin{enumerate}
    \item $p\in\{p_1,p_2,\cdots,p_n\}$且$p=p_i$,则$S^{p_1,p_2,\cdots,p_n}_{B_1,B_2,\cdots,B_n}A=B_i$,即$S^{p_1,p_2,\cdots,p_n}_{B_1,B_2,\cdots,B_n}A$为合式公式;
    \item $p\not\in\{p_1,p_2,\cdots,p_n\}$,则$S^{p_1,p_2,\cdots,p_n}_{B_1,B_2,\cdots,B_n}A=A$;
  \end{enumerate}
  \item $A=\sim B$,则$S^{p_1,p_2,\cdots,p_n}_{B_1,B_2,\cdots,B_n}B$为合式公式,所以$\sim S^{p_1,p_2,\cdots,p_n}_{B_1,B_2,\cdots,B_n}B=S^{p_1,p_2,\cdots,p_n}_{B_1,B_2,\cdots,B_n}\sim B$为合式公式,即$S^{p_1,p_2,\cdots,p_n}_{B_1,B_2,\cdots,B_n}A$为合式公式;
  \item $A=B\vee C$,则$S^{p_1,p_2,\cdots,p_n}_{B_1,B_2,\cdots,B_n}B$,$S^{p_1,p_2,\cdots,p_n}_{B_1,B_2,\cdots,B_n}C$为合式公式,所以$S^{p_1,p_2,\cdots,p_n}_{B_1,B_2,\cdots,B_n}B \vee S^{p_1,p_2,\cdots,p_n}_{B_1,B_2,\cdots,B_n}C=S^{p_1,p_2,\cdots,p_n}_{B_1,B_2,\cdots,B_n}(B\vee C)$为合式公式,即$S^{p_1,p_2,\cdots,p_n}_{B_1,B_2,\cdots,B_n}A$为合式公式;
\end{enumerate}
综上,若A是合式公式,则$S^{p_1,p_2,\cdots,p_n}_{B_1,B_2,\cdots,B_n}A$为合式公式。\newline

\noindent 9.解:
若C为永真式,根据Godel完全性定理则$\vdash C$,设C的证明序列为$C_1,C_2,\cdots,C_n$,用第二归纳法证明,$\vdash S^{p_1,p_2,\cdots,p_n}_{q_1,q_2,\cdots,q_n}C_i,1\leq i\leq n$:
\begin{enumerate}
  \item $C_i\in Aoxims$,显然$S^{p_1,p_2,\cdots,p_n}_{q_1,q_2,\cdots,q_n}C_i$也为公理
  \item 存在$j,k<i$使$C_k=C_j\supset C_i$,因为$\vdash S^{p_1,p_2,\cdots,p_n}_{q_1,q_2,\cdots,q_n}C_j$且$\vdash S^{p_1,p_2,\cdots,p_n}_{q_1,q_2,\cdots,q_n}C_k$,即$\vdash S^{p_1,p_2,\cdots,p_n}_{q_1,q_2,\cdots,q_n}C_j\supset S^{p_1,p_2,\cdots,p_n}_{q_1,q_2,\cdots,q_n}C_i$,有MP规则可推出$\vdash S^{p_1,p_2,\cdots,p_n}_{q_1,q_2,\cdots,q_n}C_i$。
\end{enumerate}
综上$\vdash S^{p_1,p_2,\cdots,p_n}_{q_1,q_2,\cdots,q_n}C$,即$\vdash D$,再由Godel完全性定理推出$\models D$,即D为永真式。\newline

\noindent 10.$ \left( \sim \left( \sim \left( q\vee r\right)\vee \sim p \right)\vee \sim \sim \left( q\vee r\right) \right)$\newline

\noindent 11.暂无

\noindent 12.解:
\begin{align*}
    1.\quad A\vee (B\vee C)&\vdash A\vee (B\vee C) \\
    2.\quad A\vee (B\vee C)&\vdash (B\vee C)\vee A\quad 1, i) \\
    3.\quad A\vee (B\vee C)&\vdash B\vee (C\vee A)\quad 2, ii) \\
    4.\quad A\vee (B\vee C)&\vdash (C\vee A) \vee B\quad 3, i) \\
    5.\quad A\vee (B\vee C)&\vdash C\vee (A \vee B)\quad 4, ii) \\
    6.\quad A\vee (B\vee C)&\vdash (A\vee B) \vee C\quad 5, i)
\end{align*}

\noindent 13.暂无

\noindent 14.可满足的

其否定对应的合取范式为$(\sim p\vee q)\wedge(\sim r\vee s)\wedge(\sim s\vee q)\wedge \sim p \wedge r$,令$S=\{\sim p\vee q,\sim r\vee s,\sim s\vee q,\sim p,r\}$,对S应用消解规则如下:
\begin{align*}
    &1.\quad S &\\
    &2.\quad \{\sim r\vee s, \sim s\vee q, r\} \quad &1,\text{关于}\sim p\text{纯文字规则} \\
    &3.\quad \{s, \sim s\vee q\} \quad &2,\text{关于}\sim r\text{单文字规则}\\
    &4.\quad \{q\} \quad &3,\text{关于}\sim s\text{单文字规则}\\
    &5.\quad \{qed\} \quad &4,\text{关于}\sim q\text{纯文字规则}\\
\end{align*}
令$\varphi(r)=f$时为真,$\varphi(p)=f,\varphi(q)=\varphi(r)=\varphi(s)=t$时为假。\newline

\noindent 15.永真式

对应的析取范式为$(\sim r\vee \sim s\vee r\vee \sim q\vee p)\wedge (q\vee \sim p\vee s\vee \sim s\vee r\vee \sim q\vee p)$,每个短句都包含互补文字,故为永真式。\newline

\noindent 16.永真式,用真值表\newline
\noindent 17.可满足,$\varphi(p)=t, \varphi(q)=f$时为真,$\varphi(p)=f, \varphi(q)=t$时为假。\newline
\noindent 18.永真式,用真值表\newline
\noindent 19.解
P'系统可以看做P系统由$\sim(p\supset q)$出发进行的证明

方法一:先证明在P系统下若$\Gamma$协调,且$\sim A\not\in Th(\Gamma)$,则$\Gamma\cup \{A\}$协调。
若$\Gamma\cup \{A\}$不协调,则存在B使得$\Gamma,A\vdash B$且$\Gamma,A\vdash \sim B$,则
\begin{align*}
  &1.\quad \Gamma,A\vdash B \quad & hyp \\
  &2.\quad \Gamma,A\vdash \sim B \quad & hyp \\
  &3.\quad \Gamma,A\vdash \sim A \quad & 1,2,DR\\
  &4.\quad \Gamma\vdash A\supset \sim A \quad&  3,CP\\
  &5.\quad \Gamma\vdash \sim A\supset \sim A  \quad & \\
  &6.\quad \Gamma\vdash \sim A \quad & 4,5,DR_3\\
\end{align*}这与$\sim A\not\in Th(\Gamma)$矛盾,故$\Gamma\cup \{A\}$协调。因为$\sim(p\supset q)$不为永真式,而P系统的定理都为永真式,故$\sim(p\supset q)\not\in Th(P)$ ,所以$Axiom\cup\{\sim(p\supset q)\}$协调,即P'系统是协调的。

方法二:假设P'系统不协调,则在P系统下存在B使得$\sim(p\supset q)\vdash B$且$\sim(p\supset q)\vdash \sim B$,根据Godel完全性定理,$\sim(p\supset q)\models B$且$\sim(p\supset q)\models \sim B$,明显矛盾,故P'系统协调。\newline

\noindent 20.解
不协调,记公理模式$A\supset B$为AS'
\begin{align*}
  &1.\quad \vdash A\vee A\supset A &AS_{1} \\
  &2.\quad \vdash (A\vee A\supset A)\supset \sim(A\vee A\supset A)&AS' \\
  &3.\quad \vdash \sim(A\vee A\supset A) &1,2,\overline{MP} \\
  &4.\quad \vdash B &3,DR \\
\end{align*}
故在P系统中增加$A\supset B$做为公理所得系统不协调。\newline

\noindent 21.解:

不存在不含“$\vee$”的定理。用反证法证明:若A为满足条件的公式,易知A中只有一个命题变元,设为p,如果辖域中包含p的“$\sim$”的个数为偶数,令指派$\varphi(p)=f$,否则$\varphi(p)=t$,则$\varphi(A)=f$,而P系统的定理都为永真式,所以P系统中不存在不含“$\vee$”的定理。\newline

\noindent 22.解:

不存在不含“$\sim$”的定理。用反证法证明:若A为满足条件的公式,其中出现的命题变元为$p_1,p_2,\cdots,p_n$,另$\varphi(p_1)=\varphi(p_2)=\cdots=\varphi(p_n)=f$,则$\varphi(A)=f$,而P系统的定理都为永真式,所以P系统中不存在不含“$\sim$”的定理。\newline

\noindent 23.解:
\begin{align*}
  &1.\quad \vdash A\vee A\supset A\vee A &Axiom \\
  &2.\quad \vdash (A\vee A\supset A\vee A) \supset \left(A\vee A\supset \sim(A\vee A)\right) &Axiom \\
  &3.\quad \vdash A\vee A\supset \sim(A\vee A) &1,2,MP \\
  &4.\quad \vdash \left(A\vee A\supset \sim(A\vee A)\right)\supset \sim(A\vee A)\vee A &Axiom \\
  &5.\quad \vdash \sim(A\vee A)\vee A &3,4,MP \\
  &6.\quad \vdash \sim(A\vee A)\vee A\supset A\vee A &Axiom \\
  &7.\quad \vdash A\vee A &5,6,MP \\
  &8.\quad \vdash A &5,7,MP \\
\end{align*}
可以看出s的公式皆为定理,故s不协调。\newline

\noindent 24.解:

A是R的定理
\begin{align*}
  &1.\quad \vdash A*A &Axiom \\
  &2.\quad \vdash A*(A*A) &Axiom \\
  &3.\quad \vdash A &1,2,<A,(B*A),B> \\
\end{align*}

\noindent 25.解:
令$\Gamma=\{\sim P\}$(P为命题变元)。
\begin{enumerate}
  \item P'是协调的
  对于P系统,$\Gamma$是可满足的,且存在唯一的指派$\varphi:{P}\rightarrow f$满足$\Gamma$,则$\Gamma\not\models p$,根据完全性定理$\Gamma\not\vdash p$,所以在P'中,命题变元p不是定理,故P'是协调的。
  \item 利用1)中的指派判断A的真值,若$\varphi(A)=t$,则A是P'的定理。构造过程为P系统下从$\Gamma$出发的证明序列。
\end{enumerate}

\noindent 26.解:
设命题变元p在A中不出现。
\begin{align*}
  &1.\quad \vdash p\supset S_{A}^p p  &Axiom \\
  &2.\quad \vdash p\supset A &1 \\
  &3.\quad \vdash p\supset A\supset S_{p\supset A}^p(p\supset A) &Axiom \\
  &4.\quad \vdash p\supset A\supset(p\supset A\supset A) &3. \\
  &5.\quad \vdash p\supset A\supset A &2,4,MP \\
  &6.\quad \vdash A &2,5,MP \\
\end{align*}
所以对任意公式A都可以构造出证明序列,所以P'不协调,但同时是完全的。\newline

\noindent 27.解:

定义$\psi$如下:$$
\left\{
  \begin{aligned}
    &\psi(p)=p\quad\text{若p为命题变元} \\
    &\psi(\wedge) = \vee \\ 
    &\psi(\vee)=\wedge \\
    &\psi(\sim) =\sim \\
    &\psi(\alpha\beta)=\psi(\alpha)\psi(\beta)
  \end{aligned}
\right.
$$
易知
\begin{itemize}
  \item $A\in L(P)$,则$\psi(A)\in L(Q)$;
  \item 若$A\in L(Q)$,则$\psi(A)\in L(P)$;
  \item $\psi(\psi(A))=A$。
\end{itemize}

首先利用第二归纳法证明必要性:假设A为Q系统的定理,且证明序列为$A_1,A_2,\cdots,A_n(=A)$,下面证明$A_i$为永假式:
\begin{itemize}
  \item $A_i\in Axiom$,显然$A_i$为永假式;
  \item 存在$j,k<i$使得$A_k=\sim A_j\wedge A_i$,根据归纳假设,因为$A_k,A_j$皆为永假式,所以$A_i$为永假式。
\end{itemize}
必要性得证。

\noindent 下面分两步进行证明充分性:
\begin{enumerate}
  \item 用第二归纳法证明:若A为P系统的定理,即$\vdash_P A$,则$\vdash_{Q}\psi(A)$,设A在P系统下的证明序列为$A_1,A_2,\cdots,A_n(=A)$
  \begin{enumerate}
    \item A为P系统的公理,易知$\psi(A)$为Q的公理,$\vdash_{Q}\psi(A)$成立
    \item 存在$j,k<i$使得$A_k=A_j\supset A_i$,则$\vdash_Q \psi(A_k)$即$\vdash_Q\sim \psi(A_j)\wedge\psi(A_i)$且$\vdash_Q \psi(A_j)$,根据Q系统的规则,可知$\vdash\psi(A_i)$成立
  \end{enumerate}
  \item 证明若$A\equiv B$,则$\psi(A)\equiv\psi(B)$
  
  若$A\equiv B$,则$\vdash_P A\equiv B$,由第一步知$\vdash_Q \psi(A\equiv B)$,即$\vdash_Q \psi(A)\equiv \sim \psi(B)$,所以$\psi(A)\equiv \sim \psi(B)$为永假式,$\psi(A)\equiv \psi(B)$为永真式
\end{enumerate}

综上,若$A\in L(Q)$且A为永假式,则$A\equiv p\wedge\sim p$,根据充分性第二步证明,则$\psi(A)\equiv \psi(p\wedge\sim p)$为永真式,则$\models_P\psi(A)$,根据完全性定理有$\vdash_P\psi(A)$,根据充分性证明的第一步有$\vdash_Q \psi(\psi(A))$,即$\vdash_QA$。充分性得证。\newline

\noindent 28.解:
用公式结构结构归纳法证明:
\begin{enumerate}
  \item A为命题变元,易知$V_\varphi(A)=V_\psi(A)$;
  \item $A=\sim B$,由归纳假设知$V_\varphi(B)=V_\psi(B)$,所以$V_\varphi(A)=\sim V_\varphi(B)=\sim V_\psi(B)=V_\psi(\sim B)=V_\psi(A)$;
  \item $A=B\vee C$,由归纳假设知$V_\varphi(B)=V_\psi(B)$且$V_\varphi(C)=V_\psi(C)$,所以$V_\varphi(A)=V_\varphi(B\vee C)=V_\varphi(B)\vee V_\varphi(C)=V_\psi(B)\vee V_\psi(C)=V_\psi(B\vee C)=V_\psi(A)$。
\end{enumerate}
综上,$V_\varphi(A)=V_\psi(A)$。\newline

\noindent 29.解:不是,参考27题

\noindent 30.解:
\begin{enumerate}
  \item \begin{align*}
    &1. \quad\vdash p\vee p\supset p &AS_1 \\
    &2. \quad\vdash A\vee A\supset A &1,sub \\
    &3. \quad\vdash p\supset q\vee p &AS_2 \\
    &4. \quad\vdash A\supset B\vee A &3,sub \\
    &5. \quad\vdash p\supset q\supset(r\vee q\supset q\vee r)&AS_3 \\
    &6. \quad\vdash A\supset B\supset(C\vee A\supset B\vee C)&5,sub \\
  \end{align*}
  由2,4,6知P的定理比为p'的定理。下面用第二归纳法证明P'的定理也为P的定理,设A为P'的定理,且证明序列为$A_1,A_2,\cdots,A_n(=A)$,则$\vdash_PA_i$
  \begin{enumerate}
    \item $A_i\in Axiom$,显然$\vdash_PA_i$成立;
    \item 存在$j,k<i$使得$A_k=A_j\supset A_i$,根据归纳假设$\vdash_PA_j$且$\vdash_PA_k$,根据P系统的MP规则知$\vdash_PA_i$成立;
    \item 存在$j<i$使得$A_i=S_{B_1,B_2,\cdots,B_n}^{p_1,p_2,\cdots,p_n}A_j$,根据归纳假设$\vdash_PA_j$,由P系统的代入规则sub知$\vdash_PA_i$成立。
  \end{enumerate}
  所以P系统和P'系统具有相同的定理。
  \item 若无sub规则,则不能推出$s\vee s\supset s$(不能推出包含除p,q,r之外命题变元的公式),故sub规则独立;若无MP规则,则不能推出$s\vee\sim s$(不能推出长度小于5的公式),故MP规则独立。
\end{enumerate}
\noindent 31.与30题重复
\noindent 32.解:

令$AS_4=\sim A\vee A$,$AS_5=A\vee\sim A$。下面证明$AS_1-AS_3$的独立性。
\begin{enumerate}
  \item 给每个命题变元以0,1,2三个可能的值,$\sim$和$\vee$的真值表定义如下:
  \begin{table}[!hbp]
    \begin{tabular}{c|c}
      & $\sim$ \\
      \hline
      0 & 1 \\
      1 & 0 \\
      2 & 2
    \end{tabular}
    \hfil
    \begin{tabular}{c|ccc}
      $\vee$ & 0 & 1 & 2 \\
      \hline
      0 & 0 & 0 & 0 \\
      1 & 0 & 1 & 2 \\
      2 & 0 & 2 & 0 \\
    \end{tabular}    
  \end{table}
  
  $AS_2-AS_5$在此数值解释下恒为0,而$AS_1$不常为0,故$AS_2-AS_5$不能表示出$AS_1$。
  \item 定义$\sim$和$\vee$的真值表定义如下:
  \begin{table}[!hbp]
    \begin{tabular}{c|c}
      & $\sim$ \\
      \hline
      0 & 2 \\
      1 & 1 \\
      2 & 0
    \end{tabular}
    \hfil
    \begin{tabular}{c|ccc}
      $\vee$ & 0 & 1 & 2 \\
      \hline
      0 & 0 & 0 & 0 \\
      1 & 0 & 1 & 2 \\
      2 & 0 & 2 & 2 \\
    \end{tabular}    
  \end{table}
  
  $AS_1,AS_3-AS_5$在此数值解释下永不为2,而$AS_2$可能为2,故$AS_1,AS_3-AS_5$不能表示出$AS_2$。
  \item 定义$\sim$和$\vee$的真值表定义如下:
  \begin{table}[!hbp]
    \begin{tabular}{c|c}
      & $\sim$ \\
      \hline
      0 & 1 \\
      1 & 2 \\
      2 & 0
    \end{tabular}
    \hfil
    \begin{tabular}{c|ccc}
      $\vee$ & 0 & 1 & 2 \\
      \hline
      0 & 0 & 0 & 0 \\
      1 & 0 & 1 & 0 \\
      2 & 0 & 0 & 2 \\
    \end{tabular}    
  \end{table}
  
  $AS_1,AS_2,AS_4,AS_5$在此数值解释下恒为0,$AS_3$不恒为0,故$AS_1,AS_2,AS_4,AS_5$不能表示出$AS_3$。
  
  综上,结论成立。
\end{enumerate}

\noindent 33.解:

构造如下的数值解释:
\begin{table}[!hbp]
  \begin{tabular}{c|c}
    & $\sim$ \\
    \hline
    0 & 1 \\
    1 & 0 \\
  \end{tabular}
  \hfil
  \begin{tabular}{c|cc}
    $\vee$ & 0 & 1  \\
    \hline
    0 & 1 & 1  \\
    1 & 0 & 1  \\
  \end{tabular}    
\end{table}

在此数值解释下,$AS_1-AS_3$恒为1,$P\supset \sim\sim P$不恒为1,且两个恒为1的公式经MP规则必得到一个恒为1的公式,故不能从$P^*$系统导出$P\supset \sim\sim P$。

\noindent 34.解:

设p,q为命题变元,A为由$\{\sim,\equiv\},p,q$构成的公式。用公式结构归纳法证明若$A_i$不是永真式或永假式,则$A_i$的真值表中取值为真的个数与取值为假的个数相等。
\begin{enumerate}
  \item $A=p$或者$A=q$,显然成立;
  \item $A=\sim B$,显然成立;
  \item $A=B\equiv C$,若$\varphi(B)=\varphi(C)$或$\varphi(B)\neq\varphi(C)$时,则$A$为永真式或永假式,当$\varphi(B),\varphi(C)$不完全一样或完全相反时($\varphi$为任意指派),通过枚举可知结论同样成立。
\end{enumerate}
所以,A为永真式或者永假式或A的真值表中t的个数与f的个数相等,而$v$的真值表中t的个数与f的个数不等,因此$\{\sim,\equiv\}$不能表示出$\vee$,则$\{\sim,\equiv\}$不完全。\newline

\noindent 35.解:

先证明对于每个n元真值函数h:$\{t,f\}^n\rightarrow\{t,f\}$,存在一个合取范式A以及n个命题变元:$p_1,p_2,\cdots,p_n$,使得$h=[\lambda p_1,\cdots\lambda p_nA]$:
\begin{enumerate}
  \item 若h恒取t,则令$A=p_1\vee\sim p_1$;
  \item 若h不恒为t,对P系统中的每个指派$\varphi$,令:
  \begin{align*}
    A^\varphi&=p_1^\varphi\vee p_2^\varphi\vee\cdots\vee p_n^\varphi \\
    p_i^\varphi&=\left\{
      \begin{aligned}
        &\sim p_i &\text{若}\varphi(p_i)=t \\
        &p_i &\text{否则}
      \end{aligned}
    \right.
  \end{align*}
  则$\varphi(A^\varphi)=\varphi(p_1^\varphi)\vee\cdots\vee\varphi(p_n^\varphi)=f$,因此$[\lambda p_1,\cdots\lambda p_nA](x_1,x_2,\cdots,x_n)=f$当且仅当$x_1=p_1^\varphi,\cdots,x_n=p_n^\varphi$。所以$A=\wedge_{h(\varphi(p_1),\varphi(p_2),\cdots,\varphi(p_n))=f}A^\varphi$满足要求。
\end{enumerate}
对于公式B,都存在一个n元真值函数$h=[\lambda p_1,\cdots\lambda p_mB]$,$p_1,p_2\cdots,p_m$为B中出现的所有命题变元,则由上述证明可知,存在一个合取范式A使得$[\lambda p_1,\cdots\lambda p_mA]=h=[\lambda p_1,\cdots\lambda p_mB]$,故$A\Leftrightarrow B$,即每个公式都有合取范式。\newline

\noindent 36.解:
$\{\vee,\wedge,\supset,\equiv\}$不是完全集

下面用公式结构归纳法证明,由命题变元p和$\{\vee,\wedge,\supset,\equiv\}$构成的公式没有永假式,也不能有$\{\vee,\wedge,\supset,\equiv\}$定义出$\sim$:
\begin{enumerate}
  \item A为命题变元p,既不是永假式,其真值表也不具备$\sim$的形式;
  \item $A=B\equiv C$或$A=B\supset C$或$A=B\vee C$或$A=B\wedge C$,由归纳假设知B和C的真值表具有以下两种形式:
  \begin{table}[!hbp]
    \begin{tabular}{c|c}
      p &  \\
      \hline
      0 & 0 \\
      1 & 1 \\
    \end{tabular}
    \hfil
    \begin{tabular}{c|c}
      p &  \\
      \hline
      0 & 1  \\
      1 & 1  \\
    \end{tabular}    
  \end{table}
\end{enumerate}
  无论$A=B\equiv C$或$A=B\supset C$或$A=B\vee C$或$A=B\wedge C$都不能使A为永假式或者具备与$\sim$相同的真值表,所以$\{\vee,\wedge,\supset,\equiv\}$表达不出$\sim$,即不完全。\newline
  
\noindent 37.解:

首先用归纳法证明:若A为仅由$\equiv$构成的合式公式(且其中的命题变元为$p_1,p_2,\cdots,p_n$),则任意改变$p_1,p_2,\cdots,p_n$的出现顺序所形成的新公式B与A等价,用$^\#A$表示A中命题变元和$\equiv$出现的次数,则$^\#A=2i+1(i=0,1,\cdots)$,
\begin{enumerate}
    \item i=0,1时显然成立;
    \item i=k\ (k>1),$A_k=A_{k-1}\equiv P_k$,设$A_{k-1}=B\equiv C$,则$A_k=B\equiv C\equiv P_k$,显然$A_k$与$C\equiv P_k \equiv B$、$P_k \equiv B \equiv C$、$B \equiv P_k  \equiv C$等价
\end{enumerate}
故仅由$\equiv$构成的命题合式公式A(且其中的命题变元为$p_1,p_2,\cdots,p_n$)与下式等价:
\begin{equation*}
    \underbrace{(p_1\equiv p_1\equiv \cdots\equiv p_1)}_{l_1}\equiv\underbrace{(p_2\equiv p_3\equiv \cdots\equiv p_2)}_{l_2}\equiv\cdots\equiv\underbrace{(p_n\equiv p_n\equiv \cdots\equiv p_n)}_{l_n}
    \end{equation*}
其中$l_i$表示$p_i$在A中出现的次数,易知若$l_i$为偶数,则该项对应与t,否则为$p_i$。所以结论成立。\newline
  
\noindent 38.略

\noindent 39.参考37

\noindent 40.解:

易知由$\wedge,\not\equiv,t,f$构成的公式只能为永真式或者永假式

用结构归纳法证明:
\begin{enumerate}
  \item A为t显然成立;
  \item $A=B\wedge C$,
  \begin{itemize}
    \item 若A能变换到1,则B和C都能变换到1,根据归纳假设,B和C都是永真式,所以A也为永真式;
    \item 若A为永真式,则B,C都是永真式,根据归纳假设,B和C都能变换到1,所以A也能变换到1;
  \end{itemize}    
  \item $A=B\not\equiv C$
  \begin{enumerate}
    \item 若A能变换到1,则B能变换到0、C能变换到1,或者B能变换到1、C能变换到0,根据归纳假设,B、C中有一个为永真式,另一个不为永真式,故A为永真式。
    \item 若A为永真式,根据归纳假设,B和C中只能有一个永真式,所以A能变换到1(根据归纳假设,B和C中的永假式只能变换到0。
  \end{enumerate}
  
\end{enumerate}
综上,结论得证。\newline

\noindent 41.解:

设C(A)中所有合适的公式的析取为B,因为C(A)的所有元素均为合取项,所以B是析取范式。
根据定理2.6.7,$\models_{\varphi}A$当且仅当有$D\in C(A)$使得$\models_\varphi D$成立。
则对任意指派$\varphi$,若$\models_{\varphi}A$,则存在$D\in C(A)$满足$\models_\varphi D$,又因为B为所有C(A)元素的析取,所以$\models_\varphi B$,必要性成立;

对任意指派$\varphi$,若$\models_{\varphi}B$,则必然存在一个B的合取项D满足$\models_\varphi D$,而$D\in C(A)$,所以$\models_{\varphi}A$,充分性成立。\newline

\noindent 42.解:

|表示与非,$p\supset q=\sim p\vee q=\sim(p\wedge\sim q)$,所以用|表示为p|(q|q)。\newline

\noindent 43.解:

| 不是可结合的,因为$f|f|t\not\equiv f|(f|t)$为永真式。\newline

\noindent 44.解:

此系统并不协调,下面给出简要的证明序列,为了减小公式长度,做以下表示:令$A_p=A|(A|A),A_r=S|Q \blacksquare A|S \blacksquare A|S$,其中$A,S,Q$均为任意公式。
\begin{align*}
  &1. \ \vdash p|(p|r)|\blacksquare p|(r|p)| \blacksquare s|q \blacksquare p|s \blacksquare p|s \\
  &2. \ \vdash A|(A|A)|\blacksquare A|(A|A)| \blacksquare S|Q \blacksquare A|S \blacksquare A|S \ &1,Rule\ i)\\
  &3. \ \vdash A_p|(A_p|A_r)|\blacksquare A_p|(A_r|A_p)| \blacksquare S|Q \blacksquare A_p|S \blacksquare A_p|S \ &1,Rule\ i)\\
  &4. \ \vdash S|Q \blacksquare A_p|S \blacksquare A_p|S \ &2,3,Rule\ ii)\\
  &5. \ \vdash S'|Q' \blacksquare A_p|S' \blacksquare A_p|S' \ &4\\
  &6. \ \vdash A_p|(S'|Q') \ &4,5,Rule\ ii)\\
  &7. \ \vdash A_p|A_p \blacksquare A_p|A_p \blacksquare A_p|A_p \ &4 \\
  &8. \ \vdash Q' \ &6,7,Rule\ ii)
\end{align*}

\noindent 45.此题可能存疑

若$S=\cup_\varphi\{p_1^\varphi\wedge p_2^\varphi\wedge\cdots\}$,其中$$p_i^\varphi=
\left\{
  \begin{aligned}
    &p_i,\quad &\text{若}\varphi(p_i)=t\\
    &\sim p_i, \quad &\text{若}\varphi(p_i)=f\\
  \end{aligned}
\right.
$$
因此,S中的每一个元素只有一个指派使其为真,所以S是析取有效的,但其任意有限子集不是析取有效的。\newline

\noindent 46.
\begin{align*}
  &1.\quad \vdash \sim A\supset B\vee \sim A \quad &\\
  &2.\quad \sim A \vdash B\vee \sim A \quad& 1,cp \\ 
  &3.\quad \sim A \vdash \sim A \vee B \quad & 2,commut\\ 
  &4.\quad \sim A \vdash A\supset B\quad & 3 \\ 
\end{align*}

\noindent 47.
\begin{align*}
  &1. \quad A\supset B,\sim(B\supset C)\supset \sim A,A \vdash A \quad &\in\\
  &2. \quad A\supset B,\sim(B\supset C)\supset \sim A,A \vdash A\supset B\quad &\in\\
  &3. \quad A\supset B,\sim(B\supset C)\supset \sim A,A \vdash B \quad &1,2,\overline{MP}\\
  &4. \quad A\supset B,\sim(B\supset C)\supset \sim A,A \vdash \sim(B\supset C)\supset \sim A \quad &\in\\
  &5. \quad A\supset B,\sim(B\supset C)\supset \sim A,A \vdash A\supset(B\supset C)\quad &4 \\
  &6. \quad A\supset B,\sim(B\supset C)\supset \sim A,A \vdash B\supset C\quad &1,5,\overline{MP} \\
  &7. \quad A\supset B,\sim(B\supset C)\supset \sim A,A \vdash C\quad &3,6,\overline{MP} \\
  &8. \quad A\supset B,\sim(B\supset C)\supset \sim A \vdash A\supset C\quad &7,CP \\
\end{align*}

\noindent 48.
\begin{align*}
  &1. \quad  \vdash B\vee B\supset B \quad &AS_1 \\
  &2. \quad  \vdash B\supset B\vee B \quad &AS_2 \\
  &3. \quad  \vdash B\vee B \quad &1,2,\overline{MP} \\
  &4. \quad  A\supset\sim(B\supset B)\vdash \sim A \vee \sim(B\supset B)\quad &\in \\
  &5. \quad  A\supset\sim(B\supset B)\vdash (B\supset B)\vee A \quad &4, commu \\
  &6. \quad  A\supset\sim(B\supset B)\vdash \sim A \quad &3,5,\overline{MP} \\
\end{align*}

\noindent 49.
\begin{align*}
  &1. \quad (A\vee B)\supset C,A \vdash A \quad & \in \\
  &2. \quad (A\vee B)\supset C,A \vdash (A\vee B)\supset C \quad & \\
  &3. \quad (A\vee B)\supset C,A \vdash A\vee B \quad &\vee_+ \\
  &4. \quad (A\vee B)\supset C,A \vdash C \quad &2,3\overline{MP} \\
  &5. \quad (A\vee B)\supset C \vdash A\vee C \quad &4,CP \\
  &6. \quad (A\vee B)\supset C \vdash B\vee C \quad \text{同理} \\
  &7. \quad (A\vee B)\supset C \vdash (A\vee C)\wedge (B\vee C)\quad &5,6,\wedge_+ \\
  &8. \quad (A\vee C)\wedge (B\vee C) \vdash (A\vee C)\wedge (B\vee C) \quad &\in \\
  &9. \quad (A\vee C)\wedge (B\vee C) \vdash (A\vee C) \quad &\wedge- \\
  &10. \quad (A\vee C)\wedge (B\vee C) \vdash (B\vee C) \quad &\wedge- \\
  &11. \quad (A\vee C)\wedge (B\vee C) \vdash (A\vee B)\supset(C\vee C) &\vee\supset\vee \\
  &12. \quad \vdash(C\vee C)\supset C \quad &AS_1 \\
  &13. \quad (A\vee C)\wedge (B\vee C) \vdash (A\vee B)\supset C &11,12,\overline{MP} \\
\end{align*}

\noindent 50.
\begin{align*}
  &1. \quad (A\wedge B)\supset C,\sim(A\supset C) \vdash A\wedge \sim C \quad &\in \\
  &2. \quad (A\wedge B)\supset C,\sim(A\supset C) \vdash A\quad &1,\wedge- \\
  &3. \quad (A\wedge B)\supset C,\sim(A\supset C) \vdash \sim C \quad &1,\wedge- \\
  &4. \quad (A\wedge B)\supset C,\sim(A\supset C) \vdash \sim C\supset \sim(A\wedge B)\quad &1,\in,\supset\sim \\
  &5. \quad (A\wedge B)\supset C,\sim(A\supset C) \vdash \sim(A\wedge B) \quad &3,4,\overline{MP} \\
  &6. \quad (A\wedge B)\supset C,\sim(A\supset C) \vdash A\supset \sim B \quad &5 \\
  &7. \quad (A\wedge B)\supset C,\sim(A\supset C) \vdash \sim B \quad &2,6,\overline{MP} \\
  &8. \quad (A\wedge B)\supset C,\sim(A\supset C) \vdash \sim B\vee C \quad &7,\vee+ \\
  &9. \quad (A\wedge B)\supset C,\sim(A\supset C) \vdash B\supset C \quad &8 \\
  &10. \quad A\vee B\supset C \vdash \sim(A\supset C)\supset(B\supset C) \quad &9,CP \\
  &11. \quad A\vee B\supset C \vdash (A\supset C)\vee (B\supset C) \quad &10 \\
  &12. \quad A\supset C,A\wedge B \vdash A\wedge B \quad &\in \\
  &13. \quad A\supset C,A\wedge B \vdash A \quad &12,\wedge- \\  
  &14. \quad A\supset C,A\wedge B \vdash A\supset C \quad &\in \\
  &15. \quad A\supset C,A\wedge B \vdash C \quad &13,14,\overline{MP} \\
  &16. \quad A\supset C \vdash A\wedge B \supset C\quad &15,CP \\
  &17. \quad B\supset C \vdash A\wedge B \supset C\quad &\text{同理} \\
  &18. \quad (A\supset C)\vee (B\supset C)\vdash  A\wedge B \supset C\quad &16,17,\vee\supset\vee \\
\end{align*}

\noindent 51.解:
\begin{align*}
  &1. \quad A\supset(B\vee C),\sim(A\supset B)\vdash A\wedge\sim B \quad&\in \\
  &2. \quad A\supset(B\vee C),\sim(A\supset B)\vdash A \quad&1,\wedge-  \\
  &3. \quad A\supset(B\vee C),\sim(A\supset B)\vdash \sim B \quad&1,\wedge- \\
  &4. \quad A\supset(B\vee C),\sim(A\supset B)\vdash A\supset(B\vee C) \quad&\in \\
  &5. \quad A\supset(B\vee C),\sim(A\supset B)\vdash B\vee C \quad& 2,4\overline{MP}\\
  &6. \quad A\supset(B\vee C),\sim(A\supset B)\vdash \sim B\supset C  \quad&5 \\
  &7. \quad A\supset(B\vee C),\sim(A\supset B)\vdash c\quad&3,6,\overline{MP} \\
  &8. \quad A\supset(B\vee C),\sim(A\supset B)\vdash \sim A\vee C \quad&7,DR_4 \\
  &9. \quad  A\supset(B\vee C),\sim(A\supset B)\vdash A\supset C \quad& 8\\
  &10. \quad A\supset(B\vee C)\vdash (A\supset B)\vee(A\supset C)\quad&9,CP \\
  &11. \quad A\supset B\vdash A\supset B\quad& \in\\
  &12. \quad A\supset B\vdash A\supset(B\vee C) \quad&11,DR_4 \\
  &13. \quad A\supset C\vdash A\supset(B\vee C) \quad&\text{同理} \\
  &14. \quad (A\supset B)\vee(A\supset C)\vdash A\supset(B\vee C) \quad& \\
\end{align*}

\noindent 52.解:
\begin{align*}
  &1. \quad A\supset(B\wedge C),A \vdash A &\quad\in \\
  &2. \quad A\supset(B\wedge C),A \vdash A\supset(B\wedge C) &\quad\in \\
  &3. \quad A\supset(B\wedge C),A \vdash B\wedge C&\quad 1,2,\overline{MP} \\
  &4. \quad A\supset(B\wedge C),A \vdash B &\quad 3,\wedge-\\
  &5. \quad A\supset(B\wedge C) \vdash A\supset B &\quad 4,CP \\
  &6. \quad A\supset(B\wedge C) \vdash A\supset C &\quad \text{同理} \\
  &7. \quad A\supset(B\wedge C) \vdash (A\supset B)\wedge(A\supset C) &\quad 5,6,\wedge+ \\
  &8. \quad A\supset B \vdash A\supset(B\vee C) &\quad DR_4 \\
  &9. \quad A\supset C \vdash A\supset(B\vee C) &\quad DR_4 \\
  &10. \quad (A\supset B)\wedge(A\supset C),A\vdash A\supset B,A\supset C, A &\quad \in \wedge- \\
  &11. \quad (A\supset B)\wedge(A\supset C),A\vdash B,C,B\wedge C &\quad 10,\overline{MP},\wedge+ \\
  &12. \quad (A\supset B)\wedge(A\supset C) \vdash A\supset(B\wedge C) &\quad 12,CP \\
\end{align*}

\noindent 53.解:
\begin{align*}
  &1. \quad (A\supset B)\wedge(B\supset A),\sim A\vee \sim B \vdash A\supset \sim B &\quad \in \\
  &2. \quad (A\supset B)\wedge(B\supset A),\sim A\vee \sim B \vdash A\supset B &\quad \in,\wedge- \\
  &3. \quad (A\supset B)\wedge(B\supset A),\sim A\vee \sim B \vdash B\supset A &\quad \in,\wedge- \\
  &4. \quad (A\supset B)\wedge(B\supset A),\sim A\vee \sim B \vdash B\supset \sim A &\quad \in \\
  &5. \quad (A\supset B)\wedge(B\supset A),\sim A\vee \sim B \vdash B\supset \sim B &\quad 1,3,\overline{MP} \\
  &6. \quad (A\supset B)\wedge(B\supset A),\sim A\vee \sim B \vdash A\supset \sim A &\quad 2,4,\overline{MP} \\
  &7. \quad (A\supset B)\wedge(B\supset A),\sim A\vee \sim B \vdash \sim A &\quad 5 \\
  &8. \quad (A\supset B)\wedge(B\supset A),\sim A\vee \sim B \vdash \sim B &\quad 6 \\
  &9. \quad (A\supset B)\wedge(B\supset A),\sim A\vee \sim B \vdash \sim A\wedge \sim B &\quad 7,8\wedge+ \\
  &10.\quad A\equiv B \vdash \sim(A\supset \sim B)\vee \sim(A\vee B) &\quad 9,CP\\
  &11.\quad A\equiv B \vdash \sim(A\equiv \sim B) &\quad 10\\
  &12.\quad \vdash (A\equiv B)\supset \sim(A\equiv \sim B) &\quad 11,CP\\
  &13.\quad \vdash (A\equiv \sim B)\supset (A\equiv B) &\quad 12,\supset\sim
\end{align*}

\noindent 54.解:
\begin{align*}
  &1.\quad A\supset(B\supset C), A\wedge B \vdash A,B &\quad \in,\wedge- \\
  &2.\quad A\supset(B\supset C), A\wedge B \vdash C, &\quad 1,\overline{MP} \\
  &3.\quad A\supset(B\supset C) \vdash (A\wedge B)\supset C, &\quad 2,CP \\
  &4.\quad (A\wedge B)\supset C,A,B \vdash A,B &\quad \in \\
  &5.\quad (A\wedge B)\supset C,A,B \vdash A\wedge B &\quad4, \wedge+ \\
  &6.\quad (A\wedge B)\supset C,A,B \vdash C &\quad5,\in,\overline{MP} \\
  &7.\quad (A\wedge B)\supset C \vdash A\supset(B\supset C) &\quad 6,MP,MP \\  
\end{align*}

\noindent 55.参考50\newline
\noindent 56.参考51\newline
\noindent 57.参考36\newline
\noindent 58.
\begin{align*}
  &\sim a = a\mid a \\ 
  &\sim a = a \downarrow a\\
  &a\vee b = (a\mid a)\mid (b\mid b) \\
  &a\vee b =(a\downarrow b)\downarrow(a\downarrow b)\\
  &a\wedge b=(a\mid b)\mid (a\mid b) \\
  &a\wedge b=(a\downarrow a)\downarrow(b\downarrow b) \\
  &a\supset b=a\mid(b\mid b) \\ 
  &a\supset b=(a\downarrow a)\downarrow b\downarrow (a\downarrow a)\downarrow b \\
  &a\equiv b = ((a\mid(b\mid b)\mid(b\mid(a\mid a)))\mid ((a\mid(b\mid b)\mid(b\mid(a\mid a))) \\
  &a\equiv b = ((a\downarrow a)\downarrow b)\downarrow((b\downarrow b)\downarrow a)
\end{align*}
对于将公式化成$\mid$或$\downarrow$的形式,可以先化成$\sim(a\wedge b)$或$\sim(a\vee b)$的形式,然后用$\mid$或$\downarrow$做替换。\newline

\noindent 59.解:
\begin{enumerate}
  \item 因为$C_1,C_2,\cdots,C_m$互异,且排序关系是全序的,所以$C_1,C_2,\cdots,C_m$排序后的结果是唯一的。
  
  令$T=\{\varphi|\varphi(A)=t\}$,因为A不是永假式,所以$T\neq \emptyset$;
  令$$x_i^\varphi=
  \left\{
    \begin{aligned}
      &x_i,\quad &\text{若}\varphi(x_i)=t\\
      &\sim x_i, \quad &\text{若}\varphi(x_i)=f\\
    \end{aligned}
  \right.
  $$
  则$\varphi(x_1^\varphi\wedge x_2^\varphi\wedge\cdots \wedge x_n^\varphi)=t$,令$C'_i=x_1^{\varphi_i}\wedge x_2^{\varphi_i}\wedge\cdots \wedge x_n^{\varphi_i}$,对$C'_1,C'_2,\cdots,C'_n$进行排序得到$C_1,C_2,\cdots,C_n$,则$B=C_1\vee C_2\vee\cdots\vee C_n$即为A的唯一的完全解析式。
  
  再证$\varphi(A)=\varphi(B)$。对于任意指派$\varphi$
  \begin{itemize}
    \item 若$\varphi(A)=t$,故存在一个$C'_i$,满足$\varphi(C'_i)=t$,又$C'_i$为B的一个合取项,故$\varphi(B)=t$。
    \item 若$\varphi(B)=t$,故B存在一个析取项C使得$\varphi(C)=t$,所以$\varphi\in T$,所以$\varphi(A)=t$.
  \end{itemize}
  \item A为永真式,则$^\#T=2^n$,故有$2^n$个短句,则$m=2^n$。
\end{enumerate}

\noindent 60.解:
\begin{itemize}
  \item 析取范式:$$(\sim p\wedge q\wedge r)\vee(p\wedge \sim q\wedge \sim r)\vee(p\wedge \sim q\wedge r)\vee(p\wedge q\wedge\sim r)$$
  合取范式:$$(p\vee q\vee r)\wedge(p\vee  q\vee \sim r)\wedge(p\vee \sim q\vee r)\wedge(\sim p\vee\sim q\vee\sim r)$$
  \item 析取范式:$$(\sim p\wedge \sim r\wedge \sim q)\vee(\sim p\wedge r\wedge q)$$
  合取范式:$$(p\vee r\vee \sim q)\wedge(p\vee\sim r\vee q)\wedge(\sim p\vee r\vee q)\wedge(\sim p\vee r\vee\sim q)\wedge(\sim p\vee \sim r\vee q)\wedge(\sim p\vee \sim r\vee \sim q)$$
  \item 析取范式:$$(\sim p\wedge \sim r\wedge \sim q)\vee(\sim p\wedge \sim r\wedge q)\vee(\sim p\wedge r\wedge q)\vee(p\wedge \sim r\wedge q)\vee(p\wedge r\wedge q)$$
  合取范式:$$(p\vee \sim r\vee q)\wedge(\sim p\vee r\vee q)\wedge(\sim p\vee \sim r\vee q)$$
\end{itemize}

\noindent 61.解:
\begin{enumerate}
  \item $(p\wedge\sim q)\vee p$
  \item $p\wedge(q\vee r)$
  \item $\sim q$
  \item $p$
\end{enumerate}

\noindent 62.解:
\begin{align*}
  &1. \quad S \quad & \\
  &2. \quad \{r,s,\sim r\vee \sim s,q\vee r\}\&\{\sim q,q\vee s,\sim r\vee \sim s,q\vee r\} \quad &\text{关于p的分裂规则} \\
  &3. \quad \{r,q,\sim r,q\vee r\}\&\{s,\sim r\vee \sim s,r\} \quad &\text{关于s和}\sim q\text{的单文字规则} \\
  &4. \quad \{\qed\}\&\{\qed\}\quad & \\
\end{align*}

\noindent 63.解:
\begin{align*}
  &1. \quad S \quad & \\
  &2. \quad \{p\vee q,p\vee\sim r,q\vee\sim r\vee p,\sim r\vee q\} \quad &1,\text{重言式规则} \\
  &3. \quad \{\sim r\vee q\} \quad &2,\text{关于p的纯文字规则} \\
  &4. \quad \emptyset \quad &3,\text{关于q的纯文字规则} \\
\end{align*}

\noindent 64.解:
\begin{align*}
  &1. \quad S \\
  &2. \quad \{\sim r,q\vee r\}\&\{q,\sim q\vee \sim r, q\vee r\} \quad &1,\text{关于p的分裂规则} \\
  &3. \quad \{q\}\&\{r,\sim r\} \quad &2,\text{关于}\sim r,q\text{的单文字规则} \\
  &4. \quad \emptyset\&\{\qed\} \quad &3,\text{关于q的纯文字规则,关于r的单文字规则} \\
\end{align*}

\noindent 65.解:
\begin{align*}
  &1. \quad S \\
  &2. \quad \{q\vee r,\sim q\vee s,\sim r\vee s, \sim s,q\vee \sim r\} \quad &1,\text{关于}\sim p\text{的单文字规则}\\
  &3. \quad \{q\vee r,\sim q,\sim r,q\vee \sim r\} \quad &2,\text{关于}\sim s\text{的单文字规则}\\
  &4. \quad \{r,\sim r\} \quad &3,\text{关于}\sim q\text{的单文字规则}\\ 
  &5. \quad \{\qed\} \quad &4,\text{关于}\sim r\text{的单文字规则}\\
\end{align*}

66-68可以先化为否定范式,画出平面图,求合取支集合,对合取支元素进行取反,然后消解,如能消解出空短句,则原式为永真式;也可以直接取反,化为析取范式,进行消解。

\noindent 66.解:
合取支集合为$\{\sim p,q\wedge \sim r,s\wedge r,s\wedge\sim q,p\wedge r,p\wedge\sim q\}$,令$S=\{p,\sim q\vee r,\sim s\vee \sim r,\sim s\vee q,\sim p\vee \sim r,\sim p\vee q\}$

对S进行消解:
\begin{align*}
  &1. \quad S \\
  &2. \quad \{\sim q\vee r,\sim s\vee \sim r, \sim s\vee q,\sim r,q\} \quad &1,\text{关于p的单文字规则}\\
  &3. \quad \{\sim q,\sim s,\sim s\vee q,q\} \quad &2,\text{关于}\sim r\text{的单文字规则}\\
  &4. \quad \{\qed\} \quad &3,\text{关于q的单文字规则,关于}\sim s\text{的纯文字规则}
\end{align*}
故原式为永真式。\newline

\noindent 67.解:
原公式取反后的短句集为$S=\{p\vee q,\sim p\vee r,\sim p\vee \sim s,\sim r\vee s,r\vee \sim q,\sim q\vee \sim s\}$

对S进行消解:
\begin{align*}
  &1. \quad S \\
  &2. \quad \{q,\sim r\vee s,r\vee \sim q,\sim q\vee \sim s\}\&\{r,\sim s,\sim r\vee s,r\vee \sim q,\sim q\vee \sim s\} \quad &1,\text{关于p的单文字规则}\\
  &3. \quad \{\sim r\vee s,r,\sim s\}\&\{\sim s,s,\sim q,\sim q\vee \sim s\} \quad &2,\text{关于q,r的单文字规则}\\
  &4. \quad \{\qed\}\&\{\qed\}
\end{align*}
故原式为永真式。\newline

\noindent 68.解:
原公式取反后短句集为$S=\{p\vee q,s\vee\sim q\vee r,\sim r\vee q,\sim s\vee q,\sim q\}$

对S进行消解:
\begin{align*}
  &1. \quad S \\
  &2. \quad \{p,\vee\sim q\vee r,\sim r,\sim s\} \quad &1,\text{关于}\sim q\text{的单文字规则}\\
  &3. \quad \{p,s\vee\sim p,\sim s\} \quad &2,\text{关于}\sim r\text{的单文字规则}\\
  &4. \quad \{s,\sim s\} \quad 3,\text{关于p的单文字规则}\\
  &5. \quad \{\qed\}
\end{align*}
故原公式为永真式。\newline

\noindent 69.略(太长,可以用平面图)

\noindent 70.解:
\begin{enumerate}
  \item P中z,Q中u
  \item z,u
  \item u,x,z
  \item g(x)对公式中u不自由
  \item h(x,y)对公式中u不自由
  \item u对公式中x自由
\end{enumerate}

\noindent 71.解:
\begin{enumerate}
  \item $\forall{x \in H}(B(x)\wedge W(t(x))\supset K(m(x)))$
  \item $\forall{xy \in H}(H(y)\wedge W(x)\wedge K(m(y))\supset \sim L(x,y))$
  \item $\sim \exists{x \in H}(B(x)\wedge W(t(x)))$
  \item 在牲口棚中有一匹马和牲口棚中所有的黑尾巴马相像
  \item 在牲口棚中不存在没有白尾巴的马
\end{enumerate}

\noindent 72.解:
\begin{enumerate}
  \item $\forall{x}(C(x,f(r))\supset U(x,r))$
  \item 如果两个人的父亲是兄弟,那么这两个人就是堂兄弟
  \item $\exists{xy}(C(x,r)\wedge B(y,r)\supset Y(x,y))$
\end{enumerate}

\noindent 73.解:
\begin{enumerate}
  \item 自由变元:x,约束变元:x,任意项都可以代入
  \item 自由变元:y,约束变元:x,y,x不出现的项可以代入
\end{enumerate}

\noindent 74.解:

n+1,n-1都有可能

\noindent 75.解:

用结构归纳法证明
\begin{enumerate}
  \item 若A为命题变元
  \begin{enumerate}
    \item A=m,则$S^{u_1,u_2.\cdots,u_n}_{x_1,x_2,\cdots,x_n}S^m_BA=S^{u_1,u_2.\cdots,u_n}_{x_1,x_2,\cdots,x_n}B$,因为$B=S_{u_1,u_2.\cdots,u_n}^{x_1,x_2,\cdots,x_n}C$且${u_1,u_2,\cdots,u_n}$不在C中出现,故$S^{u_1,u_2.\cdots,u_n}_{x_1,x_2,\cdots,x_n}B=C=S_C^mA$
    \item A为命题变元,且$A=p\neq m$, 则$S^{u_1,u_2.\cdots,u_n}_{x_1,x_2,\cdots,x_n}S^m_BA=S^{u_1,u_2.\cdots,u_n}_{x_1,x_2,\cdots,x_n}p=p=S_C^mA$
  \end{enumerate}
  \item A为原子公式,$A=P(t_1,t_2,\cdots,t_k),t_i=f(y_1,y_2,\cdots,y_j),1\leq i\leq k$,则m不在A中出现,所以 $S^{u_1,u_2.\cdots,u_n}_{x_1,x_2,\cdots,x_n}S^m_BA=A=S_C^mA$
  \item $A=\sim D$,因为$S^{u_1,u_2.\cdots,u_n}_{x_1,x_2,\cdots,x_n}S^m_BD=S_C^mD$,所以$\sim S^{u_1,u_2.\cdots,u_n}_{x_1,x_2,\cdots,x_n}S^m_BD=\sim S_C^mD$,即$S^{u_1,u_2.\cdots,u_n}_{x_1,x_2,\cdots,x_n}S^m_BA=S_C^mA$
  \item $A=D\vee E$,因为$S^{u_1,u_2.\cdots,u_n}_{x_1,x_2,\cdots,x_n}S^m_BD=S_C^mD$,且$S^{u_1,u_2.\cdots,u_n}_{x_1,x_2,\cdots,x_n}S^m_BE=S_C^mE$,则$S^{u_1,u_2.\cdots,u_n}_{x_1,x_2,\cdots,x_n}S^m_BD\vee S^{u_1,u_2.\cdots,u_n}_{x_1,x_2,\cdots,x_n}S^m_BE=S^{u_1,u_2.\cdots,u_n}_{x_1,x_2,\cdots,x_n}S^m_B( D\vee E ) =S^{u_1,u_2.\cdots,u_n}_{x_1,x_2,\cdots,x_n}S^m_BA=S_C^mD \vee S_C^mE=S_C^m(D \vee E)=S_C^mA$
  \item $A=\forall{y}D$,则$y\not\in\{u_1,u_2,\cdots,u_n\}$,则B对A中m是自由的,且$S^{u_1,u_2.\cdots,u_n}_{x_1,x_2,\cdots,x_n}S^m_BD=S_C^mD$,,所以$S^{u_1,u_2.\cdots,u_n}_{x_1,x_2,\cdots,x_n}S^m_BA=S^{u_1,u_2.\cdots,u_n}_{x_1,x_2,\cdots,x_n}S^m_B\forall{y}D=\forall{y}S^{u_1,u_2.\cdots,u_n}_{x_1,x_2,\cdots,x_n}S^m_BD=\forall{y}S_C^mD=S_C^m\forall{y}D=S_C^mA$
\end{enumerate}
综上,结论成立.\newline

\noindent 76.解:

相同公式中,运算符的运算顺序也是相同的,若$B\neq D$,假设B是D的前缀,令$D=B\vee F$,则$D\vee E=B\vee F\vee E$,故$C=F\vee E$,这就导致$B\vee C$中的“$\vee$”与$D\vee E$中的“$\vee$”运算次序不同,故B不为D的前缀,同理D不为B的前缀,从而B=D。同理得C=E。\newline
  
\noindent 77.解:

用公式结构归纳法证明:
\begin{enumerate}
  \item C为原子公式,显然y在C中不出现,$S_x^yS_y^xC=C$;
  \item $C=\sim A$,因为$S_x^yS_y^xA=A$,所以$S_x^yS_y^xC=S_x^yS_y^x\sim A=\sim S_x^yS_y^xA=\sim A=C$;
  \item $C=B\vee D$,因为$S_x^yS_y^xB=B$且$S_x^yS_y^xD=D$,所以$S_x^yS_y^xC=S_x^yS_y^x(B\vee D)=S_x^yS_y^xB\vee S_x^yS_y^xD=B\vee D=C$;
  \item $C=\forall{z}B$
  \begin{itemize}
    \item 若z=x,则$S^x_yC=S^x_y\forall{x}B=\forall{x}B=C$,y在C中不自由,$S^y_xC=C$,所以$S_x^yS_y^xC=C$;
    \item 若z=y,则$S^x_yC=S^x_y\forall{y}B$,因为y对C中x是自由的,所以x不在C中出现,故$S^x_yC=C,S^y_xC=C$,所以$S_x^yS_y^xC=C$;
    \item 若$z\neq x,y$,则$S_x^yS_y^xC=S_x^yS_y^x(\forall{z}B)=\forall{z}S_x^yS_y^xB=\forall{z}B=C$。
  \end{itemize}
\end{enumerate}
综上,若y在C中不自由,且y对C中x是自由的,则$S_x^yS_y^xC=C$。

\noindent 78.解:

令$A=(p\equiv q)\wedge(p\equiv\sim(r\supset q))\supset (q\supset\sim r)$,则A为P永真。

令$B=(\forall{x}(G(x)\supset H(x))\equiv \exists{x}H(x))\wedge (\forall{x}(G(x)\supset H(x))\equiv \sim(\forall{x}G(x)\supset \exists{x}H(x)))\supset(\exists{x}H(x)\supset\sim\forall{x}G(x))$

则$B=S^{p,q,r}_{\forall{x}(G(x)\supset H(x)),\exists{x}H(x),\forall{x}G(x)}A$,故B为P用真的,则通过P规则可以导出$\exists{x}H(x)\supset\sim\forall{x}G(x)$。\newline

\noindent 79.解:
\begin{align*}
  &1. \quad \forall{x}(R(x)\supset P(x))\wedge \forall{x}(\sim Q(x)\supset R(x)) \vdash \forall{x}(R(x)\supset P(x)),\forall{x}(\sim Q(x)\supset R(x))  \quad &\in,\wedge- \\
  &2. \quad \forall{x}(R(x)\supset P(x))\wedge \forall{x}(\sim Q(x)\supset R(x)) \vdash R(x)\supset P(x) \quad &1,AS_4\\
  &3. \quad \forall{x}(R(x)\supset P(x))\wedge \forall{x}(\sim Q(x)\supset R(x)) \vdash \sim Q(x)\supset R(x) \quad &1,AS_4\\
  &4. \quad \forall{x}(R(x)\supset P(x))\wedge \forall{x}(\sim Q(x)\supset R(x)) \vdash \sim Q(x) \supset P(x) \quad &2,3,trans\\
  &5. \quad \forall{x}(R(x)\supset P(x))\wedge \forall{x}(\sim Q(x)\supset R(x)) \vdash P(x)\vee Q(x) \quad &4.\equiv sub iii)\\
  &6. \quad \forall{x}(R(x)\supset P(x))\wedge \forall{x}(\sim Q(x)\supset R(x)) \vdash \forall{x}( P(x)\vee Q(x)) \quad &5,gen\\
  &7. \quad \vdash \forall{x}(R(x)\supset P(x))\wedge \forall{x}(\sim Q(x)\supset R(x)) \supset  \forall{x}( P(x)\vee Q(x)) \quad &6,CP
\end{align*}

\noindent 80.解:
\begin{align*}
  &1.\quad P(x)\vdash P(x) &\in \\
  &2.\quad \vdash P(x)\supset P(x) &1,CP\\
  &3.\quad \vdash S_x^y(P(x)\supset P(y)) &2\\
  &4.\quad \vdash \exists{y}(P(x)\supset P(y)) &3,\text{x对}P(x)\supset P(y)\text{中y自由}\\
  &5.\quad \vdash \forall{x}\exists{y}(P(x)\supset P(y)) &4,gen
\end{align*}

\noindent 81.解:
\begin{align*}
  &1. \quad p\supset\forall{x}\sim Q(x),p \vdash p\supset\forall{x}\sim Q(x) \quad &\in \\
  &2. \quad p\supset\forall{x}\sim Q(x),p \vdash p \quad &\in \\
  &3. \quad p\supset\forall{x}\sim Q(x),p \vdash \forall{x}\sim Q(x) \quad &1,2,\overline{MP} \\
  &4. \quad \vdash (p\supset\forall{x}\sim Q(x))\supset\forall{x}(p\supset\sim Q(x)) \quad &3,AS_4,CP,Gen,CP\\
  &5. \quad \vdash \sim\forall{x}(p\supset\sim Q(x))\supset\sim(p\supset\forall{x}\sim Q(x)) \quad &4,\supset\sim\\
  &6. \quad \vdash \exists{x}(p\wedge Q(x))\supset(p\wedge\exists{x}Q(x)) &5,\equiv sub\ iii)\\
  &7. \quad \forall{x}(p\supset\sim Q(x)),p \vdash \forall{x}(p\supset\sim Q(x)),p &\in \\
  &8. \quad \forall{x}(p\supset\sim Q(x)),p \vdash \sim Q(x) &7,AS_4,\overline{MP} \\
  &9. \quad \forall{x}(p\supset\sim Q(x))\vdash p\supset \forall{x}\sim Q(X) &8,Gen,CP\\
  &10. \quad \vdash \sim(p\supset\forall{x}\sim Q(x))\supset\sim\forall{x}(p\supset\sim Q(x)) &9,CP,\supset\sim \\
  &11. \quad \vdash p\wedge \exists{x}Q(x)\supset\exists{x}(p\wedge Q(x)) &10,\equiv sub\ iii) \\
  &12. \quad \vdash \exists{x}(p\wedge Q(x))\equiv (p\wedge\exists{x}Q(x)) &6,11
\end{align*}

\noindent 82.解:
\begin{align*}
  &1. \quad \forall{y}P(y)\vdash \forall{x}P(x) &\in,\alpha\beta \\
  &2. \quad \vdash \forall{y}P(y)\supset\forall{x}P(x) &CP \\
\end{align*}
$\forall{y}P(y)\supset\forall{x}P(x)$的前束范式为$\exists{y}\forall{x}(P(y)\supset P(x))$,根据前束范式定理,$\forall{y}P(y)\supset\forall{x}P(x)\equiv \exists{y}\forall{x}(P(y)\supset P(x))$。
所以
\begin{align*}
  &3. \quad \vdash \exists{y}\forall{x}(P(y)\supset P(x)) &2,\equiv sub\ iii)
\end{align*}
也可以演绎证明$\vdash \forall{y}P(y)\supset\forall{x}P(x)\supset \exists{y}\forall{x}(P(y)\supset P(x))$。\newline

\noindent 83.解:
\begin{align*}
  &1. \quad \forall{x}P(x)\wedge \forall{x}\sim Q(x) \vdash P(x),\sim Q(x) &\wedge-,AS_4\\
  &2. \quad \forall{x}P(x)\wedge \forall{x}\sim Q(x) \vdash \forall{x}(P(x)\wedge Q(x)) &1,\wedge+,Gen\\
  &3. \quad \vdash \exists{x}(P(x)\supset Q(x))\supset(\forall{x}P(x)\supset\exists{x}Q(x))   &2,CP,\supset\sim,\equiv sub\ iii)
  \\
  &4. \quad \forall{x}(p(x)\wedge\sim Q(x))\vdash \forall{x}P(x),\forall{x}\sim Q(x) &\in,AS_4,\wedge-,Gen\\
  &5. \quad \forall{x}(p(x)\wedge\sim Q(x))\supset \forall{x}P(x)\wedge\forall{x}\sim Q(x) &4,\wedge+, CP\\
  &6. \quad \vdash(\forall{x}P(x)\supset\exists{x}Q(x))\supset\exists{x}(P(x)\supset Q(x)) &5,\supset\sim,\equiv sub\ iii)\\
  &7. \quad \vdash \exists{x}(P(x)\supset Q(x))\equiv (\forall{x}P(x)\supset\exists{x}Q(x)) &3,6\\
\end{align*}

\noindent 84.解:
\begin{align*}
  \exists{x}\forall{y}(P(x)\equiv P(y))&\equiv\exists{x}((\sim P(x)\vee \forall{y}P(y))\wedge(\sim\exists{y}P(y)\vee P(x)))\\
  &\equiv\exists{x}((\sim P(x)\wedge\sim\exists{y}P(y))\vee(\forall{y}P(y)\wedge P(x))) \\
  &\equiv(\sim\forall{x}P(x)\wedge\sim\exists{y}P(y))\vee(\forall{y}P(y)\wedge\exists{x}P(x))\\
  &\equiv(\forall{x}P(x)\supset\forall{y}P(y))\wedge(\exists{y}P(y)\supset\forall{y}P(y))\wedge(\exists{y}P(y)\supset\exists{x}P(x))\\
  &\equiv(\exists{y}P(y)\supset\forall{y}P(y))
\end{align*}
由换名规则可得$\exists{y}P(y)\supset\forall{y}P(y)\equiv \exists{x}P(x)\supset\forall{y}P(y)$,且$\forall{y}P(y)\supset\exists{}{x}P(x)$为永真式,故
\begin{align*}
  \exists{y}P(y)\supset\forall{y}P(y) &\equiv (\exists{x}P(x)\supset\forall{y}P(y))\wedge(\forall{y}P(y)\supset\exists{}{x}P(x)) \\
  &\equiv(\exists{x}P(x)\equiv\forall{y}P(y))
\end{align*}

\noindent 85.解:

$\forall{x}(P(x)\equiv\exists{y}P(y))$同样等价于$\exists{y}P(y)\supset\forall{y}P(y)$,后续证明参考84题。\newline

\noindent 86.解:

由84题知$\exists{x}\forall{y}(P(x)\equiv P(y))\equiv(\exists{x}P(x)\supset\forall{x}P(x))$,而$(\exists{x}P(x)\supset\forall{x}P(x))\equiv(\forall{x}P(x)\vee\forall{x}\sim P(x))$

\noindent 87.略

\noindent 88.解:
\begin{enumerate}
  \item $\exists{xyz}(\sim R(x,u)\wedge(\sim R(y,z)\supset Q(u,y)))$
  \item $\exists{x}\forall{y}\exists{u}\forall{v}\sim(Q(x,y)\supset\sim(P(u,y,z)\supset\sim R(v)))$\item 
\end{enumerate}

\noindent 89.解:
\begin{enumerate}
  \item 由84题知,$\exists{y}\forall{x}(P(x)\equiv Q(y))$是$\forall{x}P(x)\equiv\exists{y}P(y)$的前束范式,另外$\exists{xy}\forall{ab}(P(x)\supset P(y)\wedge P(a)\supset P(b))$也是。
  \item $\exists{a,b,x,y}(\sim P(a,b)\supset\sim(P(x,y)\supset P(u,y)))$
\end{enumerate}

\noindent 90.解:
\begin{enumerate}
  \item 必要性
  
  因为$\models_{(I,\sigma)}\exists{x}A$,所以$I(\exists{x}A)(\sigma)=t$,即$I(\forall{x}\sim A)(\sigma)=f$,所以存在$a$使得$I(\sim A)(\sigma[x|a])=f$则$I(A)(\sigma[x|a])=t$。

  令$\varphi=\sigma[x|a]$,则满足条件$\varphi(y)=\sigma(y)\ (x\neq y)$,且$I(A)(\varphi)=t$,故$\models_{(I,\varphi)}A$。
  \item 充分性
  
  因为$\models_{(I,\varphi)}A$,所以$I(A)(\varphi)=t$,又$\varphi(y)=\sigma(y)\ (x\neq y)$,则$I(A)(\sigma[x|\varphi(x)])=t$,即$I(\sim A)(\sigma[x|\varphi(x)])=f$,故$I(\forall{x}\sim A)(\sigma)=f$,所以$I(\exists{x}A)(\sigma)=t$,即$\models_{(I,\sigma)}\exists{x}A$。
\end{enumerate}

对于91题来说,需要先将量词移到括号中去进行证明。

\noindent 91.解:
\begin{enumerate}
  \item 
  \begin{align*}
    &1.\ \forall{xz}P(z,x)\vdash\forall{z}P(z,y) \ &AS_4 \\
    &2.\ \forall{z}P(z,y)\vdash P(y,y) \ &AS_4 \\
    &3.\ \forall{xz}P(z,x)\vdash P(y,y) \ &1,2,trans\\
    &4.\ \forall{xz}P(z,x)\vdash\exists{x}\forall{y}(P(x,y)\supset P(y,y)) \ &3,\vee+,Gen,\exists +\\
    &5.\ \vdash \sim\forall{xz}P(z,x)\vee\exists{x}\forall{y}(\sim P(x,y)\vee P(y,y)) \ &4,CP\\
    &6.\ \vdash \exists{x}\forall{y}\exists{z}(\sim P(z,x)\vee(\sim P(x,y)\vee P(y,y))) \ &5,\equiv sub\ iii) \\
    &7.\ \vdash \exists{x}\forall{y}\exists{z}(P(z,x)\wedge P(x,y)\supset P(y,y)) \ &6,\equiv sub\ iii)\\
  \end{align*}
  \item 
  \begin{align*}
    &1.\ \forall{x}Q(x)\vdash \forall{y}Q(y) \ &\alpha\beta\\
    &2.\ \forall{x}Q(x)\vdash \exists{x}P(x)\vee\forall{y}Q(y) \ &2,\vee+\\
    &3.\ \vdash\sim\forall{x}Q(x)\vee(\exists{x}P(x)\vee\forall{y}Q(y)) \ &2,CP\\
    &4.\ \vdash\exists{x}\forall{y}(Q(x)\supset(P(x)\vee Q(y)))\ &3,\equiv sub\ iii)
  \end{align*}
\end{enumerate}

\noindent 92.解:$A=p\vee\sim p$
\noindent 93.解:$P(x)$
\noindent 94.解:

设$I=<D,I_0>$,$\sigma\in\sum_I$为解释I下的任意指派;

若对于任意的$a\in D$,$I(C)(\sigma[x|a])=t$,则$I(\forall{x}C)(\sigma)=t$,所以$I(C\supset\forall{x}C)(\sigma)=t$,即$C\supset\forall{x}C$在I下可满足;

若存在$a\in D$,$I(C)(\sigma[x|a])=f$,令$\varphi=\sigma[x|a]$,则$I(C)(\varphi)=f$,$I(\sim C)(\varphi)=t$,所以$I(C\supset\forall{x}C)(\varphi)=I(\sim C\vee\forall{x}C)(\varphi)=t$,即$C\supset\forall{x}C$在I下可满足;

综上,若C是任意合式公式,I是任意解释,则$C\supset\forall{x}C$在I下是可满足的。\newline

\noindent 95.解:

必要性成立,$\models A$,根据完全性定理,$\vdash A$成立,根据Gen规则有$\vdash\forall{x}A$,根据$\vee+$规则$\vdash(A\supset\forall{x}A)$,根据可靠性定理$\models(A\supset\forall{x}A)$成立。

充分性不成立,$A=p\wedge\sim p$即为反例。\newline

\noindent 96.解:
\begin{align*}
  &1.\ \forall{x}P(x)\vdash P(a) \ &AS_4\\
  &2.\ \forall{x}P(x)\vdash P(b) \ &AS_4\\
  &3.\ \forall{x}P(x)\vdash P(a)\wedge P(b) \ &1,2,\wedge+\\
  &4.\ \vdash\forall{x}P(x)\supset P(a)\wedge P(b) \ &3,CP\\
  &5.\ \vdash\exists{x}(P(x)\supset P(a)\wedge P(b)) \ &4\\
\end{align*}

对任意项t,取解释$I=<\{a,b\},I_0>$,指派$\sigma\in \sum_I$,使得$I(P(a))=t,I(P(b))=f$且$\sigma(t)=a$,则$I(P(t)\supset P(a)\wedge P(b))=f$,即不存在项t使得$\models A(t)$。\newline

\noindent 97.解:
\begin{enumerate}
  \item 可满足的。
  
  若M,N皆为永真式,$\exists{x}(M\equiv N)\supset(\exists{x}M\supset\exists{x}N)$为有效的;
  
  令$M=\forall{y}(x\times y=0),N=p\wedge\sim p$,则$\exists{M\equiv N}$为t,$\exists{x}M$为t,$\exists{x}N$为f,故$\exists{x}(M\equiv V)\supset(\exists{M}\equiv\exists{N})$为f。
  
  \item 有效的
  \begin{align*}
    &1.\ \forall{x}(M\equiv N),\forall{x}M\vdash M,M\supset N \ &\in,AS_4,\wedge- \\
    &2.\ \forall{x}(M\equiv N),\forall{x}M\vdash \forall{x}N \ &1,\overline{MP},Gen\\
    &3.\ \forall{x}(M\equiv N)\vdash \forall{x}M\supset\forall{x}N \ &2,CP\\
    &4.\ \forall{x}(M\equiv N)\vdash \forall{x}N\supset\forall{x}M \ &\text{同理}\\
    &5.\ \vdash \forall{x}(M\equiv N)\supset(\forall{x}M\equiv\forall{x}N) \ &3,4,\wedge+,CP
  \end{align*}
\end{enumerate}

\noindent 98.解:

以下的解法对题目做了稍微的修改:$A=\exists{z}(\exists{x}\forall{y}P(x,y,z)\supset\exists{W}Q(w,z))$。

\noindent A的前束范式为$\exists{z}\forall{x}\exists{x}\exists{w}(P(x,y,z)\supset Q(w,z))$\newline
B的前束范式为$\exists{z}\exists{y}\forall{x}\exists{w}(P(x,y,z)\supset Q(w,z))$\newline
C的前束范式为$\exists{z}\exists{y}\forall{x}\exists{w}(P(x,y,z)\supset Q(w,z))$\newline

故$B\equiv C,B\supset A,C\supset A$为有效的,下面对$B\supset A$进行简要的证明:

设$D=\forall{x}\exists{x}\exists{w}(P(x,y,z)\supset Q(w,z)),E=\exists{y}\forall{x}\exists{w}(P(x,y,z)\supset Q(w,z))$
\begin{align*}
  &1. \ \vdash E\supset D \\
  &2. \ \vdash \sim D\supset \sim E \ &1,\supset\sim \\
  &3. \ \forall{z}\sim D\vdash \sim D \ &AS_4 \\
  &4. \ \forall{z}\sim D\vdash \forall{z}\sim E \ &2,3,\overline{MP},Gen\\
  &5. \ \vdash \forall{z}\sim D\supset \forall{z}\sim E \ &4,CP\\
  &6. \ \vdash \exists{z}E\supset \exists{z}D \ &5,\supset\sim \\
  &7. \ \vdash B\supset A \ &6
\end{align*}

\noindent 99.解:

有效的。
\begin{align*}
  &1. \ \exists{x}P(x)\supset \forall{x}Q(x),P(x)\vdash Q(x) \ &\in,\exists+,\overline{MP},AS_4 \\
  &2. \ \vdash \exists{x}P(x)\supset \forall{x}Q(x)\supset \forall{x}(P(x)\supset Q(x)) \ &1,CP,Gen,CP
\end{align*}

\noindent 100.解:

不是有效的,令解释$I=<N,J_0>$,其中N表示自然数集,$Q(x)\text{、}P(x)$在解释I和指派$\sigma\in\sum_I$下都为“x是奇数”,显然$I(\forall{x}(P(x)\supset Q(x))\supset(\exists{x}P(x)\supset\forall{x}Q(x)))(\sigma)=f$。\newline

\noindent 101.解:

不是有效的,令解释$I=<N,J_0>$,其中N表示自然数集,$Q(x)\text{、}P(x)$在解释I和指派$\sigma\in\sum_I$下都为“x是奇数”、“x是偶数”,显然$I(\forall{x}(P(x)\supset Q(x))\equiv(\forall{x}P(x)\supset\exists{x}Q(x)))(\sigma)=f$。\newline

\noindent 102.解:

不是有效的,由84题知$\exists{x}\forall{y}(P(x)\equiv P(y))\equiv (\exists{x}P(x)\equiv\forall{y}P(y))$,显然$\exists{x}P(x)\equiv\forall{y}P(y)$不是有效的,故$\exists{x}\forall{y}(P(x)\equiv P(y))$也不是有效的。\newline

\noindent 103.解:

不是有效的,令解释$I=<N,I_0>$,其中N为自然数集,$R(x,y)$在I下表示为“$\frac{x}{y}=0$”。\newline

\noindent 104.解:
\begin{enumerate}
  \item $I(x+y=1)(\sigma[x|1][y|1])=f$,所以$I(\forall{x}\forall{y}(x+y=1))(\sigma)=f$,I'下同理。
  \item
  \begin{itemize}
    \item a为任意自然数
    \begin{align*}
      &I(x+y=1)(\sigma[x|2][y|a])=f\\
      &I(\sim (x+y=1))(\sigma[x|2][y|a])=t\\
      &I(\forall{y}\sim (x+y=1))(\sigma[x|2])=t\\
      &I(\sim\forall{y}\sim (x+y=1))(\sigma[x|2])=f\\
      &I(\forall{x}\exists{y} (x+y=1))(\sigma)=f\\
    \end{align*}
    \item a为任意实数
    \begin{align*}
      &I(x+y=1)(\sigma[x|a][y|1-a])=t\\
      &I(\sim(x+y=1))(\sigma[x|a][y|1-a])=f\\
      &I(\forall{y}\sim(x+y=1))(\sigma[x|a])=f\\
      &I(\sim\forall{y}\sim (x+y=1))(\sigma[x|a])=t\\
      &I(\forall{x}\exists{y} (x+y=1))(\sigma)=t\\
    \end{align*}
  \end{itemize}
  \item 设a为任意整数
  \begin{align*}
    &I(x+y=1)(\sigma[x|a][y|a+2])=f\\
    &I(\forall{y}(x+y=1))(\sigma[x|a])=f\\
    &I(\sim\forall{y}(x+y=1))(\sigma[x|a])=t\\
    &I(\forall{x}\sim\forall{y}(x+y=1))(\sigma)=t\\
    &I(\exists{}{x}\sim\forall{y}(x+y=1))(\sigma)=f\\
  \end{align*}
  I'下同理。
  \item 有2),3)可直接推出
  \item \begin{align*}
    &I(O(x))(\sigma[x|1])=t\\
    &I(\sim O(x))(\sigma[x|1])=f\\
    &I(\forall{x}\sim O(x))(\sigma)=f\\
    &I(\exists{x}\sim O(x))(\sigma)=t\\
    \\
    &I(O(x))(\sigma[x|2])=f\\
    &I(\forall{x} O(x))(\sigma)=f\\
  \end{align*}
  故$I(\exists{x}O(x)\supset\forall{x}O(x))(\sigma)=f$
\end{enumerate}

\noindent 105.解:

存在,$A=P(x)\vee\sim P(x)$即满足条件。\newline

\noindent 106.解:

不一定,若$A=B=P(x)$则{A,B}是协调的,若$A=\sim B=\sim P(x)$,则{A,B}不协调。\newline

\noindent 107.解:

不一定成立,$M=P,N=\sim P,C=P\wedge\sim P$。\newline

\noindent 108.解:

不一定成立,$M=N=P\vee\sim P,C=(P\vee\sim P)\wedge(\sim(P\vee\sim P))$。\newline

\noindent 110.解:

不一定成立,令$A=\sim\forall{x}P(x)$,则$A'=\sim\exists{x}P(x)$,则$\vdash A\supset A'$不成立。\newline

\noindent 111.解:定理3.7.6,也可根据定义2.4.1进行证明。\newline

\noindent 112.解:同111题。\newline

\noindent 113.解:

可由逻辑结果的定理直接推出
\begin{enumerate}
  \item 成立
  \item 不成立
\end{enumerate}

\noindent 122.解:
\begin{align*}
  &1.\ \vdash \forall{x}(A\supset B)\equiv
\end{align*}






\end{document}
