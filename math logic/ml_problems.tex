\documentclass[a4paper]{ctexart}

\usepackage[top=1in, bottom=1in, left=1in, right=1in]{geometry}
\usepackage{titlesec}
\usepackage{amsmath}
\usepackage{ulem}
\usepackage{multirow}
\usepackage{listings}
\usepackage{graphicx}
\usepackage{color}
\definecolor{grey}{RGB}{90,90,90}
\usepackage{verbatim}
\newcommand{\li}{\uline{\hspace{0.5em}}}
\newcommand{\blank}[1]{(\emph{\underline{#1}})}


\begin{document}

\begin{enumerate}
  
  \item $P_{46}$ 紧致性在下述情况下如何解释:
  $\{\sim(p_1^{\varphi_1}\wedge p_2^{\varphi_1}\wedge\cdots),\sim(p_1^{\varphi_2}\wedge p_2^{\varphi_2}\wedge\cdots),\cdots\}$是不可满足的,但其任何有限子集是可满足的。

  \item $P_{73}$中$A_N^M$表示用N替换M在A中的所有指定出现之结果,并由此推导出了\textit{定理 3.2.3},这意味该定理中所有的$A_N^M$均为替换所有指定出现。$P_{76}$中推导\textit{定理 3.2.4}的过程用到了\textit{定理 3.2.3},但其中的替换是部分替换,因此\textit{定理 3.2.3}的证明过程是否正确,应该如何解释?
  
  \item $P_{73}$CP定理的证明,同上一问,$\alpha\beta$规则按照定义是对所有的指定出现进行替换,但\textit{定理 3.3.4}第五步证明过程中的$\alpha\beta$规则只是进行了部分替换。
  
  \item $P_{106}$辖域,$A\wedge B$出现在几个命题连接词的辖域中?(2014.二.3)
  
  \item $P_{117}$ 第n步的计算真值完全可以用第l步和第m步的结果,为什么还要搞一个$\sigma[x|b][x|a]$?
  
  \item $P_{139}$ 保守扩张下$L(F_1)\subseteq L(F_2)$,膨胀和扩张有何关系?
  
  \item $P_{150}$ 若公理为永假式,且为系统调的,那么就没有模型。
  
  
  \blankspace{3}
  \item $Q_{13}$ 如何解题?   
  
  \item 44题是否存误,Q系统并不协调。
  
  \item $Q_{45}$ $\{p_1^{\varphi_1}\wedge p_2^{\varphi_1}\wedge\cdots,p_1^{\varphi_2}\wedge p_2^{\varphi_2}\wedge\cdots,\cdots\}$是析取有效的,但其任何有限子集不是析取有效的。
   
  \item $Q_{74}$ 长度不唯一?
  
  \item 76题,解法。
   
  \item 90题,y能否代表非个体变元? 
  
  \item 134题到136题
   
  \item 137题:若两个集合都是无限的如何解?
   
  \item 138题是否存误?若解释$I=<\{a\},I_0>$,$I_0(P)(a,a)=t$,则I的论域是又穷的,且公式是可满足的。
  
  \blankspace{3}
  
  \item 2015.四题目存误?若x不在$\Gamma$中出现,$x\not\in\{x_1,x_2,\cdots,x_n\}$,但$x=y_i$且$x_i$在$\Gamma$中有自由出现,那么$\delta(x)=x$在$\Gamma$有自由出现。
  
  \item 2014.一.4:是否不是短句集就不能消解?
  
  \item 2012.一.h):题目打印错误?
  
  \item 2009.一.6: $\forall{x}\forall{y}P(x,y)$
  
  \item 2008.六:给出具体的证明还是只给出一个例子即可?
    
  \item 2007.一
  
  \item 2005.二.c 不是永真式    
   
  
  \item 2004.七:什么是初等等价?
   
  
  \item 2003.五:
  
  \item 2001.2 2002.3 完备的
  
  \item 一个系统是否是完全的是什么意思?
  
  
  \blankspace{3}
  \item 等词及后续章节会不会有考点?
  
  \item 有些关于新系统的题目中只给了几个命题连接词,但又要用到额外的命题连接词,按照默认的定义?
  

\end{enumerate}
  
\end{document}
