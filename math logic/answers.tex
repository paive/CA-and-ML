\documentclass[a4paper]{ctexart}

\usepackage[top=1in, bottom=1in, left=1in, right=1in]{geometry}
\usepackage{titlesec}
\usepackage{amsmath}
\usepackage{multirow}
\usepackage{ulem}
\newcommand{\li}{\uline{\hspace{0.5em}}}
\renewcommand{\labelenumi}{\arabic{enumi})}

\begin{document}

\title{数理逻辑课后题/考题答案}
\author{作者:杨森}
\maketitle
\newpage

\section{课后题}
\noindent 1.略

\noindent 2.解:
\begin{enumerate}
  \item =>
  利用结构归纳法证明
  \begin{enumerate}
    \item Y是命题变元,此时Y的生成序列即为自身;
    \item $Y=\neg A$,A的生成序列为$A_1,A_2,\cdots,A_m(=A)$,则Y的生成序列为$A_1,A_2,\cdots,A_m,Y$;
    \item $Y=B\vee C$,B的生成序列为$B_1,B_2,\cdots,B_m(=B)$,C的生成序列为$C_1,C_2,\cdots,C_n(=C)$,则Y的生成序列为$B_1,B_2,\cdots,B_m,C_1,C_2,\cdots,C_n,Y=(B\vee C)$
  \end{enumerate}
  
  \item <= 
  利用第二归纳法证明
  
  假设Y的生成序列为$Y_1,Y_2,\cdots,Y_m(=Y)$,证明$Y_i(1\leq i\leq m)$是合式公式
  \begin{enumerate}
    \item $Y_i$是命题变元,则$Y_i$是合适公式;
    \item $Y_i=\neg Y_j(j<i)$,因为$Y_j$是合式公式,故$Y_i$也是合式公式;
    \item $Y_i=Y_j \vee Y_k(j,k<i)$,因为$Y_j$,$Y_k$是合式公式,故$Y_i$也是合式公式。
  \end{enumerate}
\end{enumerate}
综上,Y为合式公式当且仅当Y有一个生成序列。\newline
\noindent 3.略(根据公式定义进行证明)

\noindent 4.略

\noindent 5.解:
用结构归纳法证明
\begin{enumerate}
  \item A为命题变元p,显然结论成立;
  \item $A=\neg B$,因为B满足条件,则$\neg (B)$即A也满足条件;
  \item $A=B\vee C$,因为B,C满足条件,则$(B)\vee (C)$也满足条件
\end{enumerate}
综上,若表达式A为合式公式,则最终计数为0.\newline
\noindent 6.略

\noindent 7.略

\noindent 8.解:
用公式结构归纳法证明
\begin{enumerate}
  \item A为命题变元p
  \begin{enumerate}
    \item $p\in\{p_1,p_2,\cdots,p_n\}$且$p=p_i$,则$S^{p_1,p_2,\cdots,p_n}_{B_1,B_2,\cdots,B_n}A=B_i$,即$S^{p_1,p_2,\cdots,p_n}_{B_1,B_2,\cdots,B_n}A$为合式公式;
    \item $p\not\in\{p_1,p_2,\cdots,p_n\}$,则$S^{p_1,p_2,\cdots,p_n}_{B_1,B_2,\cdots,B_n}A=A$;
  \end{enumerate}
  \item $A=\neg B$,则$S^{p_1,p_2,\cdots,p_n}_{B_1,B_2,\cdots,B_n}B$为合式公式,所以$\neg S^{p_1,p_2,\cdots,p_n}_{B_1,B_2,\cdots,B_n}B=S^{p_1,p_2,\cdots,p_n}_{B_1,B_2,\cdots,B_n}\neg B$为合式公式,即$S^{p_1,p_2,\cdots,p_n}_{B_1,B_2,\cdots,B_n}A$为合式公式;
  \item $A=B\vee C$,则$S^{p_1,p_2,\cdots,p_n}_{B_1,B_2,\cdots,B_n}B$,$S^{p_1,p_2,\cdots,p_n}_{B_1,B_2,\cdots,B_n}C$为合式公式,所以$S^{p_1,p_2,\cdots,p_n}_{B_1,B_2,\cdots,B_n}B \vee S^{p_1,p_2,\cdots,p_n}_{B_1,B_2,\cdots,B_n}C=S^{p_1,p_2,\cdots,p_n}_{B_1,B_2,\cdots,B_n}(B\vee C)$为合式公式,即$S^{p_1,p_2,\cdots,p_n}_{B_1,B_2,\cdots,B_n}A$为合式公式;
\end{enumerate}
综上,若A是合式公式,则$S^{p_1,p_2,\cdots,p_n}_{B_1,B_2,\cdots,B_n}A$为合式公式。\newline

\noindent 9.解:
若C为永真式,根据Godel完全性定理则$\vdash C$,设C的证明序列为$C_1,C_2,\cdots,C_n$,用第二归纳法证明,$\vdash S^{p_1,p_2,\cdots,p_n}_{q_1,q_2,\cdots,q_n}C_i,1\leq i\leq n$:
\begin{enumerate}
  \item $C_i\in Aoxims$,显然$S^{p_1,p_2,\cdots,p_n}_{q_1,q_2,\cdots,q_n}C_i$也为公理
  \item 存在$j,k<i$使$C_k=C_j\supset C_i$,因为$\vdash S^{p_1,p_2,\cdots,p_n}_{q_1,q_2,\cdots,q_n}C_j$且$\vdash S^{p_1,p_2,\cdots,p_n}_{q_1,q_2,\cdots,q_n}C_k$,即$\vdash S^{p_1,p_2,\cdots,p_n}_{q_1,q_2,\cdots,q_n}C_j\supset S^{p_1,p_2,\cdots,p_n}_{q_1,q_2,\cdots,q_n}C_i$,有MP规则可推出$\vdash S^{p_1,p_2,\cdots,p_n}_{q_1,q_2,\cdots,q_n}C_i$。
\end{enumerate}
综上$\vdash S^{p_1,p_2,\cdots,p_n}_{q_1,q_2,\cdots,q_n}C$,即$\vdash D$,再由Godel完全性定理推出$\models D$,即D为永真式。\newline

\noindent 10.$ \left( \neg \left( \neg \left( q\vee r\right)\vee \neg p \right)\vee \neg \neg \left( q\vee r\right) \right)$\newline

\noindent 12.解:
\begin{align*}
    1.\quad A\vee (B\vee C)&\vdash A\vee (B\vee C) \\
    2.\quad A\vee (B\vee C)&\vdash (B\vee C)\vee A\quad 1, i) \\
    3.\quad A\vee (B\vee C)&\vdash B\vee (C\vee A)\quad 2, ii) \\
    4.\quad A\vee (B\vee C)&\vdash (C\vee A) \vee B\quad 3, i) \\
    5.\quad A\vee (B\vee C)&\vdash C\vee (A \vee B)\quad 4, ii) \\
    6.\quad A\vee (B\vee C)&\vdash (A\vee B) \vee C\quad 5, i)
\end{align*}

\noindent 13.暂无

\noindent 14.可满足的

其否定对应的合取范式为$(\neg p\vee q)\wedge(\neg r\vee s)\wedge(\neg s\vee q)\wedge \neg p \wedge r$,令$S=\{\neg p\vee q,\neg r\vee s,\neg s\vee q,\neg p,r\}$,对S应用消解规则如下:
\begin{align*}
    &1.\quad S &\\
    &2.\quad \{\neg r\vee s, \neg s\vee q, r\} \quad &1,\text{关于}\neg p\text{纯文字规则} \\
    &3.\quad \{s, \neg s\vee q\} \quad &2,\text{关于}\neg r\text{单文字规则}\\
    &4.\quad \{q\} \quad &3,\text{关于}\neg s\text{单文字规则}\\
    &5.\quad \{qed\} \quad &4,\text{关于}\neg q\text{纯文字规则}\\
\end{align*}
令$\varphi(r)=f$时为真,$\varphi(p)=f,\varphi(q)=\varphi(r)=\varphi(s)=t$时为假。\newline

\noindent 15.永真式

对应的析取范式为$(\neg r\vee \neg s\vee r\vee \neg q\vee p)\wedge (q\vee \neg p\vee s\vee \neg s\vee r\vee \neg q\vee p)$,每个短句都包含互补文字,故为永真式。\newline

\noindent 16.永真式,用真值表\newline
\noindent 17.可满足,$\varphi(p)=t, \varphi(q)=f$时为真,$\varphi(p)=f, \varphi(q)=t$时为假。\newline
\noindent 18.永真式,用真值表\newline
\noindent 19.解
P'系统可以看做P系统由$\neg(p\supset q)$出发进行的证明

方法一:先证明在P系统下若$\Gamma$不协调,且$\neg A\not\in Th(\Gamma)$,则$\Gamma\cup \{A\}$协调。
若$\Gamma\cup \{A\}$不协调,则存在B使得$\Gamma,A\vdash B$且$\Gamma,A\vdash \neg B$,则
\begin{align*}
  &1.\quad \Gamma,A\vdash B \quad & hyp \\
  &2.\quad \Gamma,A\vdash \neg B \quad & hyp \\
  &3.\quad \Gamma,A\vdash \neg A \quad & 1,2,DR\\
  &4.\quad \Gamma\vdash A\supset \neg A \quad&  3,CP\\
  &5.\quad \Gamma\vdash \neg A\supset \neg A  \quad & \\
  &6.\quad \Gamma\vdash \neg A \quad & 4,5,DR_3\\
\end{align*}这与$\neg A\not\in Th(\Gamma)$矛盾,故$\Gamma\cup \{A\}$协调。因为$\neg(p\supset q)$不为永真式,而P系统的定理都为永真式,故$\neg(p\supset q)\not\in Th(P)$ ,所以$Axiom\cup\{\neg(p\supset q)\}$协调,即P'系统是协调的。

方法二:假设P'系统不协调,则在P系统下存在B使得$\neg(p\supset q)\vdash B$且$\neg(p\supset q)\vdash \neg B$,根据Godel完全性定理,$\neg(p\supset q)\models B$且$\neg(p\supset q)\models \neg B$,明显矛盾,故P'系统协调。\newline

\noindent 20.解
不协调,记公理模式$A\supset B$为AS'
\begin{align*}
  &1.\quad \vdash A\vee A\supset A &AS_{1} \\
  &2.\quad \vdash (A\vee A\supset A)\supset \neg(A\vee A\supset A)&AS' \\
  &3.\quad \vdash \neg(A\vee A\supset A) &1,2,\bar{MP} \\
  &4.\quad \vdash B &3,DR \\
\end{align*}
故在P系统中增加$A\supset B$做为公理所得系统不协调。\newline

\noindent 21.解:

不存在不含“$\vee$”的定理。用反证法证明:若A为满足条件的公式,易知A中只有一个命题变元,设为p,如果辖域中包含p的“$\neg$”的个数为偶数,令指派$\varphi(p)=f$,否则$\varphi(p)=t$,则$\varphi(A)=f$,而P系统的定理都为永真式,所以P系统中不存在不含“$\vee$”的定理。\newline

\noindent 22.解:

不存在不含“$\neg$”的定理。用反证法证明:若A为满足条件的公式,其中出现的命题变元为$p_1,p_2,\cdots,p_n$,另$\varphi(p_1)=\varphi(p_2)=\cdots=\varphi(p_n)=f$,则$\varphi(A)=f$,而P系统的定理都为永真式,所以P系统中不存在不含“$\neg$”的定理。\newline

\noindent 23.解:
\begin{align*}
  &1.\quad \vdash A\vee A\supset A\vee A &Axiom \\
  &2.\quad \vdash (A\vee A\supset A\vee A) \supset \left(A\vee A\supset \neg(A\vee A)\right) &Axiom \\
  &3.\quad \vdash A\vee A\supset \neg(A\vee A) &1,2,MP \\
  &4.\quad \vdash \left(A\vee A\supset \neg(A\vee A)\right)\supset \neg(A\vee A)\vee A &Axiom \\
  &5.\quad \vdash \neg(A\vee A)\vee A &3,4,MP \\
  &6.\quad \vdash \neg(A\vee A)\vee A\supset A\vee A &Axiom \\
  &7.\quad \vdash A\vee A &5,6,MP \\
  &8.\quad \vdash A &5,7,MP \\
\end{align*}
可以看出s的公式皆为定理,故s不协调。\newline

\noindent 24.解:

A是R的定理
\begin{align*}
  &1.\quad \vdash A*A &Axiom \\
  &2.\quad \vdash A*(A*A) &Axiom \\
  &3.\quad \vdash A &1,2,<A,(B*A),B> \\
\end{align*}

\noindent 25.解:
令$\Gamma=\{\neg P\}$(P为命题变元)。
\begin{enumerate}
  \item P'是协调的
  对于P系统,$\Gamma$是可满足的,且存在唯一的指派$\varphi:{P}\rightarrow f$满足$\Gamma$,则$\Gamma\not\models p$,根据完全性定理$\Gamma\not\vdash p$,所以在P'中,命题变元p不是定理,故P'是协调的。
  \item 利用1)中的指派判断A的真值,若$\varphi(A)=t$,则A是P'的定理。构造过程为P系统下从$\Gamma$出发的证明序列。
\end{enumerate}

\noindent 26.解:
设命题变元p在A中不出现。
\begin{align*}
  &1.\quad \vdash p\supset S_{A}^p p  &Axiom \\
  &2.\quad \vdash p\supset A &1 \\
  &3.\quad \vdash p\supset A\supset S_{p\supset A}^p(p\supset A) &Axiom \\
  &4.\quad \vdash p\supset A\supset(p\supset A\supset A) &3. \\
  &5.\quad \vdash p\supset A\supset A &2,4,MP \\
  &6.\quad \vdash A &2,5,MP \\
\end{align*}
所以对任意公式A都可以构造出证明序列,所以P'不协调,但同时是完全的。\newline

\noindent 27.解:

定义$\psi$如下:$$
\left\{
  \begin{aligned}
    &\psi(p)=p\quad\text{若p为命题变元} \\
    &\psi(\wedge) = \vee \\ 
    &\psi(\vee)=\wedge \\
    &\psi(\neg) =\neg \\
    &\psi(\alpha\beta)=\psi(\alpha)\psi(\beta)
  \end{aligned}
\right.
$$
易知
\begin{itemize}
  \item $A\in L(P)$,则$\psi(A)\in L(Q)$;
  \item 若$A\in L(Q)$,则$\psi(A)\in L(P)$;
  \item $\psi(\psi(A))=A$。
\end{itemize}

首先利用第二归纳法证明必要性:假设A为Q系统的定理,且证明序列为$A_1,A_2,\cdots,A_n(=A)$,下面证明$A_i$为永假式:
\begin{itemize}
  \item $A_i\in Axiom$,显然$A_i$为永假式;
  \item 存在$j,k<i$使得$A_k=\neg A_j\wedge A_i$,根据归纳假设,因为$A_k,A_j$皆为永假式,所以$A_i$为永假式。
\end{itemize}
必要性得证。

\noindent 下面分两步进行证明充分性:
\begin{enumerate}
  \item 用第二归纳法证明,若A为P系统的定理,则$\psi(A)$为Q系统定理,设A在P系统下的证明序列为$A_1,A_2,\cdots,A_n(=A)$:
  \begin{enumerate}
    \item $A_i\in Axiom_P$,显然$A_i\in Axiom_Q$,则$\vdash_Q\psi(A_i)$;
    \item 存在$j,k<i$,使得$A_k=A_j\supset A_i$即$A_k=\neg A_j\vee A_i$,则根据归纳假设$\vdash \psi(A_j)$且$\vdash \psi(\neg A_j\vee A_i)$即$\vdash \neg\psi(A_j)\wedge \psi(A_i)$,所以$\vdash\psi(A_i)$。
  \end{enumerate}
  \item 用结构归纳法证明:若$A\in L(Q)$且A为永假式,则$\psi(A)\in L(P)$且A为永真式,若$A\in L(Q)$为永真式,则$\psi(A)\in L(P)$为永假式。
  \begin{enumerate}
    \item $A=p\wedge\neg p$或者$A=\neg p\wedge p$,显然A为Q系统的最简永假式,$\psi(A)\in L(P)$为永真式;
    \item $A=\neg(p\wedge\neg p)$或者$A=\neg(\neg p\wedge p)$,显然A为Q系统的最简永真式,$\psi(A)\in L(P)$为永假式;
    \item $A=\neg B$且A为永假式,根据归纳假设,B为永真式,且$\psi(B)\in L(P)$且为永假式,因为$\psi(A)=\psi(\neg B)=\neg\psi(B)$,则$\psi(A)\in L(P)$为永真式;
    \item $A=\neg B$且A为永真式,根据归纳假设,B为永假式,且$\psi(B)\in L(P)$且为永真式,因为$\psi(A)=\psi(\neg B)=\neg\psi(B)$,则$\psi(A)\in L(P)$为永假式;
    \item $A=B\wedge C$且A为永假式,则存在以下三种情况:
    \begin{itemize}
      \item B为永假式,根据归纳假设$\psi(B)\in L(P)$且为永真式,因为$\psi(A)=\psi(B\wedge C)=\psi(B)\vee \psi(C)$,则$\psi(A)\in L(P)$为永真式;
      \item C为永假式,证明同上;
      \item B,C均不为永假式,但对任意指派$\varphi$都有$\varphi(B)\neq \varphi(C)$,又因为$\psi(A)=\psi(B\wedge C)=\psi(B)\vee \psi(C)$,所以$\psi(A)\in L(P)$为永真式;
    \end{itemize}
    \item $A=B\wedge C$且A为永真式,则B和C皆为永真式,根据归纳假设,$\psi(B)\in L(P)$且为永假式且$\psi(C)\in L(P)$且为永假式,又因为又因为$\psi(A)=\psi(B\wedge C)=\psi(B)\vee \psi(C)$,故$\psi(A)\in L(P)$为永假式。
  \end{enumerate}
\end{enumerate}
综上,若$A\in L(Q)$且A为永假式,根据充分性第二步证明,则$\psi(A)\in L(P)$且为永真式,即$\models_P\psi(A)$,根据完全性定理有$\vdash_P\psi(A)$,根据充分性证明的第一步有$\vdash_Q \psi(\psi(A))$,即$\vdash_QA$。充分性得证。\newline

\noindent 28.解:
用公式结构结构归纳法证明:
\begin{enumerate}
  \item A为命题变元,易知$V_\varphi(A)=V_\psi(A)$;
  \item $A=\neg B$,由归纳假设知$V_\varphi(B)=V_\psi(B)$,所以$V_\varphi(A)=\neg V_\varphi(B)=\neg V_\psi(B)=V_\psi(\neg B)=V_\psi(A)$;
  \item $A=B\vee C$,由归纳假设知$V_\varphi(B)=V_\psi(B)$且$V_\varphi(C)=V_\psi(C)$,所以$V_\varphi(A)=V_\varphi(B\vee C)=V_\varphi(B)\vee V_\varphi(C)=V_\psi(B)\vee V_\psi(C)=V_\psi(B\vee C)=V_\psi(A)$。
\end{enumerate}
综上,$V_\varphi(A)=V_\psi(A)$。\newline

\noindent 29.解:不是,参考27题

\noindent 30.解:
\begin{enumerate}
  \item \begin{align*}
    &1. \quad\vdash p\vee p\supset p &AS_1 \\
    &2. \quad\vdash A\vee A\supset A &1,sub \\
    &3. \quad\vdash p\supset q\vee p &AS_2 \\
    &4. \quad\vdash A\supset B\vee A &3,sub \\
    &5. \quad\vdash p\supset q\supset(r\vee q\supset q\vee r)&AS_3 \\
    &6. \quad\vdash A\supset B\supset(C\vee A\supset B\vee C)&5,sub \\
  \end{align*}
  由2,4,6知P的定理比为p'的定理。下面用第二归纳法证明P'的定理也为P的定理,设A为P'的定理,且证明序列为$A_1,A_2,\cdots,A_n(=A)$,则$\vdash_PA_i$
  \begin{enumerate}
    \item $A_i\in Axiom$,显然$\vdash_PA_i$成立;
    \item 存在$j,k<i$使得$A_k=A_j\supset A_i$,根据归纳假设$\vdash_PA_j$且$\vdash_PA_k$,根据P系统的MP规则知$\vdash_PA_i$成立;
    \item 存在$j<i$使得$A_i=S_{B_1,B_2,\cdots,B_n}^{p_1,p_2,\cdots,p_n}A_j$,根据归纳假设$\vdash_PA_j$,由P系统的代入规则sub知$\vdash_PA_i$成立。
  \end{enumerate}
  所以P系统和P'系统具有相同的定理。
  \item 若无sub规则,则不能推出$s\vee s\supset s$(不能推出包含除p,q,r之外命题变元的公式),故sub规则独立;若无MP规则,则不能推出$s\vee\neg s$(不能推出长度小于5的公式),故MP规则独立。
\end{enumerate}
\noindent 31.与30题重复
\noindent 32.解:

令$AS_4=\neg A\vee A$,$AS_5=A\vee\neg A$。下面证明$AS_1-AS_3$的独立性。
\begin{enumerate}
  \item 给每个命题变元以0,1,2三个可能的值,$\neg$和$\vee$的真值表定义如下:
  \begin{table}[!hbp]
    \begin{tabular}{c|c}
      & $\neg$ \\
      \hline
      0 & 1 \\
      1 & 0 \\
      2 & 2
    \end{tabular}
    \hfil
    \begin{tabular}{c|ccc}
      $\vee$ & 0 & 1 & 2 \\
      \hline
      0 & 0 & 0 & 0 \\
      1 & 0 & 1 & 2 \\
      2 & 0 & 2 & 0 \\
    \end{tabular}    
  \end{table}
  
  $AS_2-AS_5$在此数值解释下恒为0,而$AS_1$不常为0,故$AS_2-AS_5$不能表示出$AS_1$。
  \item 定义$\neg$和$\vee$的真值表定义如下:
  \begin{table}[!hbp]
    \begin{tabular}{c|c}
      & $\neg$ \\
      \hline
      0 & 2 \\
      1 & 1 \\
      2 & 0
    \end{tabular}
    \hfil
    \begin{tabular}{c|ccc}
      $\vee$ & 0 & 1 & 2 \\
      \hline
      0 & 0 & 0 & 0 \\
      1 & 0 & 1 & 2 \\
      2 & 0 & 2 & 2 \\
    \end{tabular}    
  \end{table}
  
  $AS_1,AS_3-AS_5$在此数值解释下永不为2,而$AS_2$可能为2,故$AS_1,AS_3-AS_5$不能表示出$AS_2$。
  \item 定义$\neg$和$\vee$的真值表定义如下:
  \begin{table}[!hbp]
    \begin{tabular}{c|c}
      & $\neg$ \\
      \hline
      0 & 1 \\
      1 & 2 \\
      2 & 0
    \end{tabular}
    \hfil
    \begin{tabular}{c|ccc}
      $\vee$ & 0 & 1 & 2 \\
      \hline
      0 & 0 & 0 & 0 \\
      1 & 0 & 1 & 0 \\
      2 & 0 & 0 & 2 \\
    \end{tabular}    
  \end{table}
  
  $AS_1,AS_2,AS_4,AS_5$在此数值解释下恒为0,$AS_3$不恒为0,故$AS_1,AS_2,AS_4,AS_5$不能表示出$AS_3$。
  
  综上,结论成立。
\end{enumerate}

\noindent 33.解:

构造如下的数值解释:
\begin{table}[!hbp]
  \begin{tabular}{c|c}
    & $\neg$ \\
    \hline
    0 & 1 \\
    1 & 0 \\
  \end{tabular}
  \hfil
  \begin{tabular}{c|cc}
    $\vee$ & 0 & 1  \\
    \hline
    0 & 1 & 1  \\
    1 & 0 & 1  \\
  \end{tabular}    
\end{table}

在此数值解释下,$AS_1-AS_3$恒为1,$P\supset \neg\neg P$不恒为1,且两个恒为1的公式经MP规则必得到一个恒为1的公式,故不能从$P^*$系统导出$P\supset \neg\neg P$。

\noindent 34.解:

设p,q为命题变元,A为由$\{\neg,\equiv\},p,q$构成的公式。用公式结构归纳法证明若$A_i$不是永真式或永假式,则$A_i$的真值表中取值为真的个数与取值为假的个数相等。
\begin{enumerate}
  \item $A=p$或者$A=q$,显然成立;
  \item $A=\neg B$,显然成立;
  \item $A=B\equiv C$,若$\varphi(B)=\varphi(C)$或$\varphi(B)\neq\varphi(C)$时,则$A$为永真式或永假式,当$\varphi(B),\varphi(C)$不完全一样或完全相反时($\varphi$为任意指派),通过枚举可知结论同样成立。
\end{enumerate}
所以,A为永真式或者永假式或A的真值表中t的个数与f的个数相等,而$v$的真值表中t的个数与f的个数不等,因此$\{\neg,\equiv\}$不能表示出$\vee$,则$\{\neg,\equiv\}$不完全。\newline

\noindent 35.解:

先证明对于每个n元真值函数h:$\{t,f\}^n\rightarrow\{t,f\}$,存在一个合取范式A以及n个命题变元:$p_1,p_2,\cdots,p_n$,使得$h=[\lambda p_1,\cdots\lambda p_nA]$:
\begin{enumerate}
  \item 若h恒取t,则令$A=p_1\vee\neg p_1$;
  \item 若h不恒为t,对P系统中的每个指派$\varphi$,令:
  \begin{align*}
    A^\varphi&=p_1^\varphi\vee p_2^\varphi\vee\cdots\vee p_n^\varphi \\
    p_i^\varphi&=\left\{
      \begin{aligned}
        &\neg p_i &\text{若}\varphi(p_i)=t \\
        &p_i &\text{否则}
      \end{aligned}
    \right.
  \end{align*}
  则$\varphi(A^\varphi)=\varphi(p_1^\varphi)\vee\cdots\vee\varphi(p_n^\varphi)=f$,因此$[\lambda p_1,\cdots\lambda p_nA](x_1,x_2,\cdots,x_n)=f$当且仅当$x_1=p_1^\varphi,\cdots,x_n=p_n^\varphi$。所以$A=\wedge_{h(\varphi(p_1),\varphi(p_2),\cdots,\varphi(p_n))=f}A^\varphi$满足要求。
\end{enumerate}
对于公式B,都存在一个n元真值函数$h=[\lambda p_1,\cdots\lambda p_mB]$,$p_1,p_2\cdots,p_m$为B中出现的所有命题变元,则由上述证明可知,存在一个合取范式A使得$[\lambda p_1,\cdots\lambda p_mA]=h=[\lambda p_1,\cdots\lambda p_mB]$,故$A\Leftrightarrow B$,即每个公式都有合取范式。\newline

\noindent 36.解:
$\{\vee,\wedge,\supset,\equiv\}$不是完全集

下面用公式结构归纳法证明,由命题变元p和$\{\vee,\wedge,\supset,\equiv\}$构成的公式没有永假式,也不能有$\{\vee,\wedge,\supset,\equiv\}$定义出$\neg$:
\begin{enumerate}
  \item 对于命题变元P既不是永假式,其真值表也不具备$\neg$的形式;
\end{enumerate}

\noindent 75.解:
用结构归纳法证明
\begin{enumerate}
  \item 若A为命题变元
  \begin{enumerate}
    \item A=m,则$S^{u_1,u_2.\cdots,u_n}_{x_1,x_2,\cdots,x_n}S^m_BA=S^{u_1,u_2.\cdots,u_n}_{x_1,x_2,\cdots,x_n}B$,因为$B=S_{u_1,u_2.\cdots,u_n}^{x_1,x_2,\cdots,x_n}C$且${u_1,u_2,\cdots,u_n}$不在C中出现,故$S^{u_1,u_2.\cdots,u_n}_{x_1,x_2,\cdots,x_n}B=C=S_C^mA$
    \item A为命题变元,且$A=p\neq m$, 则$S^{u_1,u_2.\cdots,u_n}_{x_1,x_2,\cdots,x_n}S^m_BA=S^{u_1,u_2.\cdots,u_n}_{x_1,x_2,\cdots,x_n}p=p=S_C^mA$
  \end{enumerate}
  \item A为原子公式,$A=P(t_1,t_2,\cdots,t_k),t_i=f(y_1,y_2,\cdots,y_j),1\leq i\leq k$,则m不在A中出现,所以 $S^{u_1,u_2.\cdots,u_n}_{x_1,x_2,\cdots,x_n}S^m_BA=A=S_C^mA$
  \item $A=\neg D$,因为$S^{u_1,u_2.\cdots,u_n}_{x_1,x_2,\cdots,x_n}S^m_BD=S_C^mD$,所以$\neg S^{u_1,u_2.\cdots,u_n}_{x_1,x_2,\cdots,x_n}S^m_BD=\neg S_C^mD$,即$S^{u_1,u_2.\cdots,u_n}_{x_1,x_2,\cdots,x_n}S^m_BA=S_C^mA$
  \item $A=D\vee E$,因为$S^{u_1,u_2.\cdots,u_n}_{x_1,x_2,\cdots,x_n}S^m_BD=S_C^mD$,且$S^{u_1,u_2.\cdots,u_n}_{x_1,x_2,\cdots,x_n}S^m_BE=S_C^mE$,则$S^{u_1,u_2.\cdots,u_n}_{x_1,x_2,\cdots,x_n}S^m_BD\vee S^{u_1,u_2.\cdots,u_n}_{x_1,x_2,\cdots,x_n}S^m_BE=S^{u_1,u_2.\cdots,u_n}_{x_1,x_2,\cdots,x_n}S^m_B( D\vee E ) =S^{u_1,u_2.\cdots,u_n}_{x_1,x_2,\cdots,x_n}S^m_BA=S_C^mD \vee S_C^mE=S_C^m(D \vee E)=S_C^mA$
  \item $A=\forall{y}D$,则$y\not\in\{u_1,u_2,\cdots,u_n\}$,则B对A中m是自由的,且$S^{u_1,u_2.\cdots,u_n}_{x_1,x_2,\cdots,x_n}S^m_BD=S_C^mD$,,所以$S^{u_1,u_2.\cdots,u_n}_{x_1,x_2,\cdots,x_n}S^m_BA=S^{u_1,u_2.\cdots,u_n}_{x_1,x_2,\cdots,x_n}S^m_B\forall{y}D=\forall{y}S^{u_1,u_2.\cdots,u_n}_{x_1,x_2,\cdots,x_n}S^m_BD=\forall{y}S_C^mD=S_C^m\forall{y}D=S_C^mA$
\end{enumerate}
综上,结论成立.\newline

\noindent 76.解:
  
  



\end{document}
